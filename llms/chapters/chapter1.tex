\chapter{Estructura algebraicas y matemáticas antes de los números}

En esta sección, estudiamos las estructuras pre-numéricas que son fundamentales para entender la inteligencia
artificial. Estructuras como los conjuntos las funciones nos permiten describir matemáticamente colecciones de objetos,
las conexiones entre ellos y las operaciones que se pueden realizar sobre ellos. Un enfoque clave está en los grupos,
que se utilizan para modelar las transformaciones de los datos y las simetrías en los modelos de aprendizaje
automático.

\section{Conjuntos, mapas y funciones}

A primera vista, los números pueden parecer los objetos más elementales en matemáticas. Sin embargo, es posible
identificar estructuras aún más simples y básicas. De hecho, los números pueden sumarse, restarse, multiplicarse, etc.,
lo cual requiere un conjunto de reglas que definan cómo se realizan estas operaciones. ¿Pero qué pasa si consideramos
solo una colección de objetos, despojados de cualquier suposición adicional sobre ellos?

\begin{definition}
    Un \textit{conjunto} es una colección de objetos distintos, llamados \textit{elementos} o \textit{miembros} del conjunto.
\end{definition}

Estos elementos pueden ser cualquier cosa: números, símbolos o incluso otros conjuntos. Lo que caracteriza a un
conjunto es que no permite la repetición de elementos (es decir, cada elemento aparece solo una vez en un conjunto), y
el orden en que aparecen los elementos no importa (es decir, los conjuntos no están ordenados). Los conjuntos son la
base para definir estructuras matemáticas más complejas.

Típicamente se denotan con letras mayúsculas, como $A, B, X$, etc. Los miembros de un conjunto se listan dentro de
llaves $\{\;\}$, y si un elemento $x$ pertenece a un conjunto $A$, escribimos $x \in A$, que se lee como «$x$ es un
elemento de $A$». Si $x$ no pertenece a $A$, escribimos $x \notin A$. Por ejemplo, si $A = \{1, 2, 3\}$, entonces $2
    \in A$, pero $4 \notin A$. Algunos ejemplos son:

\begin{itemize}
    \item El \textbf{conjunto vacío}, un conjunto sin elementos. Se denota por $\emptyset$ o a veces por $\{\;\}$.
    \item El \textbf{conjunto Unitario (Singleton)}, un conjunto con exactamente un elemento, por ejemplo, $\{1\}$.
    \item El \textbf{conjunto de los números naturales}. Denotado por $\NN$, es el conjunto de todos los enteros no negativos:
          $\NN = \{0, 1, 2, 3, \dots\}$.
    \item El \textbf{conjunto de los enteros}, que incluye números positivos, números negativos y el cero.
    \item El \textbf{conjunto de los números racionales}, que son números que pueden expresarse como una razón de dos enteros.
    \item El \textbf{conjunto de los números reales}, incluyendo tanto números racionales (e.g., $1, 0.75, -3$) como números
          irracionales (e.g., $\pi, \sqrt{2}$).
    \item El \textbf{conjunto de los números complejos}, que pueden escribirse como $a + bi$, donde $a$ y $b$ son números reales
          e $i$ es la unidad imaginaria con $i^2 = -1$.
\end{itemize}

\subsection{Notación de conjuntos y operaciones}

\begin{itemize}
    \item \textbf{Notación constructiva de conjuntos}: Se utiliza para describir un conjunto especificando una expresión o la forma general de un elemento, seguida de una barra vertical separadora $|$, y a su derecha, una regla que la expresión de la izquierda debe satisfacer.
          \[
              \{x \mid f(x)\} = \{\text{expresión} \mid \text{regla satisfecha por la expresión}\}.
          \]
          En palabras, se puede leer como «$x$ tal que (para el cual) $f(x)$».

    \item \textbf{Subconjunto}: Un conjunto $A$ es un subconjunto de un conjunto $B$, escrito $A \subseteq B$, si cada elemento de $A$ es también un elemento de $B$. Si $A \subseteq B$ pero $A \neq B$, decimos que $A$ es un \textbf{subconjunto propio}, escrito $A \subset B$.

    \item \textbf{Unión}: La unión de dos conjuntos $A$ y $B$, escrita $A \cup B$, es el conjunto de todos los elementos que están en $A$, en $B$, o en ambos.

    \item \textbf{Intersección}: La intersección de dos conjuntos $A$ y $B$, escrita $A \cap B$, es el conjunto de todos los elementos que están tanto en $A$ como en $B$.
    \item \textbf{Diferencia}: La diferencia de dos conjuntos $A$ y $B$, escrita $A \setminus B$, es el conjunto de todos los elementos que están en $A$ pero no en $B$.

    \item \textbf{Complemento}: El complemento de un conjunto $A$, escrito $A^c$, es el conjunto de todos los elementos que
          no están en $A$, asumiendo que existe un conjunto universal $U$ que contiene todos los elementos bajo consideración.

    \item \textbf{Conjunto potencia}: El conjunto potencia de un conjunto $A$, denotado por $\mathcal{P}(A)$, es el conjunto de todos los subconjuntos de $A$, incluyendo el conjunto vacío y el propio $A$.

    \item \textbf{Cardinalidad}: La cardinalidad de un conjunto es el tamaño o número de elementos que contiene. Si un conjunto es finito, su cardinalidad es un entero no negativo. Para conjuntos infinitos, la cardinalidad se define de manera más abstracta: dos conjuntos infinitos tienen la misma cardinalidad si existe una biyección entre sus elementos. La cardinalidad de un conjunto $A$ se denota por $|A|$ o a veces $\#(A)$.
\end{itemize}

Hay múltiples formas de especificar un conjunto, además de listar sus elementos explícitamente. Por ejemplo, podemos
definir el conjunto de los números naturales pares como
\[
    \{2x \mid x \in \mathbb{N}\} = \{x \in \mathbb{N} \mid x \text{ es par}\} = \{2, 4, 6, 8, \dots\}.
\]
Alternativamente, a veces la regla que deben satisfacer los elementos del conjunto podría ser una ecuación:
\[
    \{x \in \mathbb{Z} \mid x > 0\} = \mathbb{N},
\]
\[
    \{x \in \mathbb{Q} \mid x^2 = 2\} = \emptyset.
\]
En el último ejemplo, las soluciones a la ecuación $x^2 = 2$ son las raíces $x = \pm\sqrt{2}$, que son números
irracionales y, por lo tanto, no son elementos de $\mathbb{Q}$. Así, la regla no tiene elementos que la satisfagan, lo
que significa que hemos encontrado una complicada de describir el conjunto vacío.

\subsection*{Conjuntos finitos y operaciones simples}
Consideremos los conjuntos finitos $B = \{1, 2, 3, 4, 5\}$, $A = \{1, 2, 3\}$, $C = \{1, 2, 3, 4, 5\}$, entonces $A
    \subset B$ y $C\subseteq B$, esto porque $A \neq B$, mientras que $C = B$. Sus cardinalidades serían $|A| = 3$, $|B| =
    5$, y $|C| = 5$. Las uniones e intersecciones en este ejemplo son $C \cup B = C \cap B = C = B$, $A \cup B = B$, y $A
    \cap B = A$. Otro ejemplo interesante es la cardinalidad del conjunto vacío $|\emptyset| = 0$ y la cardinalidad del
conjunto unitario que contiene el conjunto vacío $|\{\emptyset\}| = 1$.

\subsection*{Conjuntos infinitos y operaciones simples}
Considere los conjuntos infinitos $\mathbb{N} = \{1, 2, 3, 4, 5, \dots\}$ y $\mathbb{E} = \{2, 4, 6, 8, \dots\}$, el
conjunto de los números naturales y los números naturales pares, respectivamente. No es sorprendente que $\mathbb{E}
    \subset \mathbb{N}$ ya que cada elemento de $\mathbb{E}$ es un elemento de $\mathbb{N}$. Sin embargo, a diferencia de
los conjuntos finitos, las cardinalidades de $\mathbb{N}$ y $\mathbb{E}$ son iguales, denotado como $|\mathbb{N}| =
    |\mathbb{E}| = \aleph_0$. Esto se debe al hecho de que existe una \textit{biyección} entre $\mathbb{N}$ y $\mathbb{E}$
(explicaremos las biyecciones con más detalle pronto). Una biyección $f: \mathbb{N} \to \mathbb{E}$ puede definirse
como $f(n) = 2n$. Para cada número natural $n \in \mathbb{N}$, $f(n)$ produce un elemento único de $\mathbb{E}$, y cada
elemento de $\mathbb{E}$ es alcanzado exactamente una vez. Por ejemplo: $f(1)=2, f(2)=4, f(3)=6, \dots$. Por lo tanto,
a pesar de que $\mathbb{E}$ es un subconjunto propio de $\mathbb{N}$, su cardinalidad infinita sigue siendo la misma.

En términos de otras operaciones: $\mathbb{E} \cup \mathbb{N} = \mathbb{N}$, $\mathbb{E} \cap \mathbb{N} = \mathbb{E}$
y $\mathbb{N} \setminus \mathbb{E} = \{1, 3, 5, 7, \dots\}$. Notablemente, la cardinalidad del conjunto que contiene,
por ejemplo, los conjuntos infinitos $\mathbb{R}$ y $\mathbb{N}$ es en realidad $|\{\mathbb{R}, \mathbb{N}\}| = 2$, ya
que el conjunto solo contiene dos elementos, a pesar de que los elementos en sí mismos son infinitos.

\begin{tcolorbox}[colback=orange!10!white,colframe=orange!50!black,title=Conjuntos en Aprendizaje Profundo Geométrico y Redes Neuronales de Grafos]
    En Aprendizaje Profundo Geométrico, a menudo estamos interesados en modelar señales en colecciones de nodos, aristas y parches en una variedad, por ejemplo. Como veremos más adelante en la Sección 7, en el contexto de las GNNs, el dominio geométrico se define como un grafo $G = (V, E)$, que es una tupla que consiste en un conjunto de nodos $V$ y un conjunto de aristas $E$. De manera similar, para modelar la vecindad de un nodo, se utilizan conjuntos múltiples (conjuntos que permiten la repetición de elementos).
\end{tcolorbox}

\section*{Productos cartesianos}
Después de introducir conjuntos y algunas operaciones básicas, definamos el producto cartesiano. Aunque el concepto puede parecer abstracto inicialmente, juega un papel importante al discutir variedades y construir espacios más complejos combinando elementos de subespacios más simples. El producto cartesiano se utiliza para modelar sistemas compuestos y relaciones entre elementos de dos o más conjuntos.

\begin{tcolorbox}[colback=gray!10!white,colframe=gray!50!black,title=Definición]
    El \textit{producto cartesiano} de dos conjuntos $A$ y $B$, denotado por $A \times B$, es el conjunto de todos los pares ordenados $(a, b)$ donde $a \in A$ y $b \in B$:
    \[
        A \times B = \{(a, b) \mid a \in A, b \in B\}.
    \]
\end{tcolorbox}

Por ejemplo, sea $A = \{1, 2\}$ y $B = \{b_1, b_2\}$. Su producto $A \times B$ es:
\[
    A \times B = \{(1, b_1), (1, b_2), (2, b_1), (2, b_2)\}.
\]
