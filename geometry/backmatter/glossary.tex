% Definición de entradas del glosario relacionadas con Álgebra Lineal
\newglossaryentry{matriz}{
    name={Matriz},
    description={Arreglo rectangular de elementos organizados en filas y columnas. Se utiliza para representar transformaciones lineales, sistemas de ecuaciones y conjuntos de datos}
}

\newglossaryentry{matriz-cuadrada}{
    name={Matriz cuadrada},
    description={Matriz que tiene el mismo número de filas que columnas}
}

\newglossaryentry{matriz-identidad}{
    name={Matriz identidad},
    description={Matriz cuadrada con unos en la diagonal principal y ceros en el resto de posiciones. Actúa como elemento neutro en la multiplicación de matrices}
}

\newglossaryentry{matriz-transpuesta}{
    name={Matriz transpuesta},
    description={Matriz que resulta de intercambiar las filas y columnas de una matriz original}
}

\newglossaryentry{matriz-simetrica}{
    name={Matriz simétrica},
    description={Matriz cuadrada que es igual a su transpuesta. Sus elementos cumplen que $a_{ij} = a_{ji}$ para todo $i,j$}
}

\newglossaryentry{determinante}{
    name={Determinante},
    description={Función escalar asociada a una matriz cuadrada. Proporciona información sobre si la matriz es invertible y sobre el factor de cambio de volumen en transformaciones lineales}
}

\newglossaryentry{matriz-inversa}{
    name={Matriz inversa},
    description={Para una matriz cuadrada $A$, es la matriz $A^{-1}$ tal que $AA^{-1} = A^{-1}A = I$, donde $I$ es la matriz identidad}
}

\newglossaryentry{traza}{
    name={Traza},
    description={Suma de los elementos de la diagonal principal de una matriz cuadrada}
}

\newglossaryentry{matriz-ortogonal}{
    name={Matriz ortogonal},
    description={Matriz cuadrada cuya inversa coincide con su transpuesta. Las matrices ortogonales preservan el producto escalar y representan rotaciones y reflexiones}
}

\newglossaryentry{matriz-adjunta}{
    name={Matriz adjunta},
    description={Matriz cuya entrada $(i,j)$ es el cofactor de la entrada $(j,i)$ de la matriz original}
}

\newglossaryentry{vector}{
    name={Vector},
    description={Elemento de un espacio vectorial. En el contexto de álgebra lineal, generalmente se representa como una matriz columna o fila}
}

\newglossaryentry{rango}{
    name={Rango},
    description={El número máximo de vectores linealmente independientes en las filas o columnas de una matriz. Equivalente a la dimensión del espacio imagen de la transformación lineal representada por la matriz}
}

\newglossaryentry{autovector}{
    name={Autovector},
    description={Vector no nulo $v$ que, al ser transformado por una matriz $A$, resulta en un múltiplo escalar de sí mismo: $Av = \lambda v$, donde $\lambda$ es el autovalor correspondiente}
}

\newglossaryentry{autovalor}{
    name={Autovalor},
    description={Valor escalar $\lambda$ asociado a un autovector $v$ de una matriz $A$, que satisface la ecuación $Av = \lambda v$}
}

\newglossaryentry{matriz-diagonal}{
    name={Matriz diagonal},
    description={Matriz cuadrada en la que todos los elementos fuera de la diagonal principal son cero}
}

\newglossaryentry{matriz-antisimetrica}{
    name={Matriz antisimétrica},
    description={Matriz cuadrada que es igual al negativo de su transpuesta. Sus elementos cumplen que $a_{ij} = -a_{ji}$ para todo $i,j$}
}

\newglossaryentry{matriz-unitaria}{
    name={Matriz unitaria},
    description={Matriz compleja cuya inversa coincide con su traspuesta conjugada. Las matrices unitarias preservan el producto escalar hermítico y son la generalización de las matrices ortogonales en espacios complejos}
}
