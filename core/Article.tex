%%% CLASE DEL DOCUMENTO
\documentclass[letterpaper, 12pt]{article}

%%% CODIFICACIÓN Y FUENTES FUNDAMENTALES
\usepackage[T1]{fontenc}
\usepackage[utf8]{inputenc}
\usepackage{lmodern}
\usepackage{microtype}
\DisableLigatures{encoding = *, family = * }

%%% IDIOMA Y LOCALIZACIÓN
\usepackage[spanish, es-noshorthands]{babel}

%%% GEOMETRÍA DEL DOCUMENTO
\usepackage[top=3cm, bottom=2cm, right=2cm, left=2cm, headheight=15pt]{geometry}
\usepackage{multicol}
\usepackage{lscape}

%%% MATEMÁTICAS UNIFICADAS
\usepackage{amsmath}
\usepackage{amsthm}
\usepackage{amsfonts}
\usepackage{amssymb}
\usepackage{mathtools}

%%% COLORES Y GRÁFICOS
\usepackage[dvipsnames]{xcolor}
\usepackage{graphicx}
\graphicspath{{./images/}}
\DeclareGraphicsExtensions{.pdf,.png,.jpg,.jpeg}

%%% DEFINICIÓN DE COLORES PARA ENTORNOS MATEMÁTICOS
\definecolor{theoremColor}{RGB}{15,76,129}     % Azul oscuro
\definecolor{definitionColor}{RGB}{49,99,159}  % Azul medio
\definecolor{exampleColor}{RGB}{46,139,87}     % Verde oscuro
\definecolor{remarkColor}{RGB}{169,81,81}      % Rojo suave
\definecolor{proofColor}{RGB}{102,102,102}     % Gris
\definecolor{boxColor}{RGB}{245,247,250}       % Fondo claro
\definecolor{borderColor}{RGB}{230,236,245}    % Borde suave

%%% PAQUETES PARA ESTILO DE TEOREMAS
\usepackage{mdframed}
\usepackage{thmtools}
\usepackage{shadethm}

%%% HIPERVÍNCULOS Y REFERENCIAS
\usepackage[
    breaklinks=true,
    colorlinks=true,
    linkcolor=blue,
    citecolor=blue,
    urlcolor=gray
]{hyperref}
\usepackage{xurl}
\usepackage{ccicons}

%%% TABLAS MEJORADAS
\usepackage{array}
\usepackage{longtable}
\usepackage{booktabs}
\usepackage{multirow}
\usepackage{colortbl}

%%% CÓDIGO Y ALGORITMOS
\usepackage{listings}
\usepackage[spanish,onelanguage,ruled,vlined]{algorithm2e}
\usepackage[noend]{algpseudocode}

%%% BIBLIOGRAFÍA
\usepackage{natbib}
\usepackage{apalike}

%%% ÍNDICES Y REFERENCIAS CRUZADAS
\usepackage{imakeidx}
\makeindex[columns=3, title=Indice]

%%% CONFIGURACIÓN DE CÓDIGO
\renewcommand{\lstlistingname}{Código}
\lstset{ 
    keywordstyle=\color{blue},
    basicstyle=\footnotesize\ttfamily,
    commentstyle=\color{green!60!black}\ttfamily,
    stringstyle=\color{purple},
    backgroundcolor=\color{gray!5},
    rulecolor=\color{gray!5},
    upquote=true,
    numbers=left,
    numberstyle=\tiny\color{gray},
    stepnumber=1,
    numbersep=8pt,
    showstringspaces=false,
    breaklines=true,
    frameround=ftff,
    frame=single,
    belowcaptionskip=-1.4em,
    belowskip=0em,
    aboveskip=-1.4em,
    captionpos=b,
    escapeinside={\%*}{*)},
    literate={á}{{\'a}}1 {ã}{{\~a}}1 {é}{{\'e}}1 {í}{{\'i}}1 {ú}{{\'u}}1 {ó}{{\'o}}1
}

%%% CONFIGURACIÓN DE ESTILOS DE TEOREMAS
\mdfsetup{
    skipabove=\topskip,
    skipbelow=\topskip,
    innertopmargin=10pt,
    innerbottommargin=10pt,
    innerrightmargin=10pt,
    innerleftmargin=10pt,
    backgroundcolor=boxColor,
    roundcorner=3pt,
    linewidth=0.5pt,
}

%%% ESTILOS DE TEOREMAS
\declaretheoremstyle[
    headfont=\normalfont\bfseries\color{theoremColor},
    bodyfont=\normalfont,
    headpunct={.},
    mdframed={
        linecolor=theoremColor,
        backgroundcolor=boxColor,
        linewidth=1pt,
        roundcorner=3pt,
        skipabove=\topskip,
        skipbelow=\topskip
    }
]{theoremstyle}

\declaretheoremstyle[
    headfont=\normalfont\bfseries\color{definitionColor},
    bodyfont=\normalfont,
    headpunct={.},
    mdframed={
        linecolor=definitionColor,
        backgroundcolor=boxColor,
        linewidth=1pt,
        roundcorner=3pt,
        skipabove=\topskip,
        skipbelow=\topskip
    }
]{definitionstyle}

\declaretheoremstyle[
    headfont=\normalfont\bfseries\color{exampleColor},
    bodyfont=\normalfont,
    headpunct={.},
    mdframed={
        linecolor=exampleColor,
        backgroundcolor=boxColor,
        linewidth=1pt,
        roundcorner=3pt,
        skipabove=\topskip,
        skipbelow=\topskip
    }
]{examplestyle}

\declaretheoremstyle[
    headfont=\normalfont\bfseries\color{remarkColor},
    bodyfont=\normalfont,
    headpunct={.},
    mdframed={
        linecolor=remarkColor,
        backgroundcolor=boxColor,
        linewidth=1pt,
        roundcorner=3pt,
        skipabove=\topskip,
        skipbelow=\topskip
    }
]{remarkstyle}

%%% DECLARACIÓN DE TEOREMAS
\declaretheorem[style=theoremstyle, name=Teorema, numberwithin=section]{theorem}
\declaretheorem[style=theoremstyle, name=Lema, sibling=theorem]{lemma}
\declaretheorem[style=theoremstyle, name=Corolario, sibling=theorem]{corollary}
\declaretheorem[style=definitionstyle, name=Definición, numberwithin=section]{definition}
\declaretheorem[style=examplestyle, name=Ejemplo, numberwithin=section]{example}
\declaretheorem[style=remarkstyle, name=Observación, numbered=no]{remark}

%%% ESTILO PARA DEMOSTRACIONES
\renewenvironment{proof}[1][\proofname]{\par
    \pushQED{\qed}%
    \normalfont \topsep6\p@\@plus6\p@\relax
    \trivlist
    \item[\hskip\labelsep
    {\color{proofColor}\itshape
        #1\@addpunct{.}}]\ignorespaces
    }{\popQED\endtrivlist\@endpefalse}

%%% COMANDOS MATEMÁTICOS
\usepackage{xargs}
% Conjuntos numéricos
\newcommand{\RR}{\mathbb{R}}
\newcommand{\ZZ}{\mathbb{Z}}
\newcommand{\NN}{\mathbb{N}}
\newcommand{\QQ}{\mathbb{Q}}
\newcommand{\CC}{\mathbb{C}}
\newcommand{\FF}{\mathbb{F}}
\newcommand{\KK}{\mathbb{K}}

% Comandos especiales
\newcommandx{\points}[3][1=0, 3=n]{#2_{_{#1}},\ldots,#2_{_{#3}}} 

% Ecuaciones destacadas
\newcommand{\boxedeq}[1]{%
    \begin{mdframed}[
        linecolor=theoremColor,
        backgroundcolor=boxColor,
        linewidth=0.5pt,
        roundcorner=3pt
    ]
    \begin{equation}
    #1
    \end{equation}
    \end{mdframed}
}

%%% OPERADORES MATEMÁTICOS
\DeclareMathOperator{\sen}{sen}
\DeclareMathOperator{\rot}{rot}
\DeclareMathOperator{\dive}{div}
\DeclareMathOperator{\im}{img}
\DeclareMathOperator{\id}{id}
\DeclareMathOperator{\Tr}{Tr}
\DeclareMathOperator{\argMin}{arg\,min}

%%% CONFIGURACIÓN DE PÁGINA
\usepackage{fancyhdr}
\pagestyle{fancy}
\renewcommand{\sectionmark}[1]{\markboth{#1}{}}
\fancyhf{}
\fancyhead[RO]{\nouppercase{\small\textit\leftmark \,/}{\small\textit \thepage}}
\renewcommand{\headrulewidth}{0pt}

%%% ESPACIADO Y PÁRRAFOS
\setlength{\parindent}{0cm}
\setlength{\parskip}{2mm}
\linespread{1.2}
\setlength{\marginparwidth}{2cm}

%%% CONTROL DE LÍNEAS HUÉRFANAS Y VIUDAS
\clubpenalty=1000
\widowpenalty=1000
\tolerance=1414
\hbadness=1414

%%% NOTAS Y COMENTARIOS
\usepackage[colorinlistoftodos, prependcaption]{todonotes}
\newcommand{\note}[2][]{%
    \todo[linecolor=green,backgroundcolor=green!25,bordercolor=green,#1]{#2}%
}

%%% CONFIGURACIÓN DE MÁRGENES PERSONALIZADOS
\newenvironment{changemargin}[2]{%
    \list{}{\rightmargin#2\leftmargin#1}\item[]}{\endlist}

%%% AJUSTE DE ESPACIO ENTRE TEOREMA Y DEMOSTRACIÓN
\usepackage{etoolbox}
\AtBeginEnvironment{proof}{\vspace{-1.0\baselineskip}}