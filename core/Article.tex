% & -shell-escape

%%% CLASE DEL DOCUMENTO
\documentclass[letterpaper, 12pt]{article}

%%% PAQUETES BASE
\usepackage[T1]{fontenc}        
\usepackage[utf8]{inputenc}    
\usepackage{lmodern}
\usepackage[spanish, es-noshorthands]{babel}
\usepackage{comment}
\usepackage{textcomp}

%%% GEOMETRÍA DEL DOCUMENTO
\usepackage[top=3cm, bottom=2cm, right=2cm, left=2cm, headheight=15pt]{geometry}
\usepackage{multicol}
\usepackage{lscape}

%%% PAQUETES MATEMÁTICOS
\usepackage{amsmath, amsthm, amsfonts, latexsym, cancel}
\usepackage{amsbsy}
\usepackage{thmtools}
\usepackage{amssymb}
\usepackage{mathtools}

%%% PAQUETES PARA ESTILOS Y DISEÑO
\usepackage[dvipsnames]{xcolor}
\usepackage{tikz}
\usetikzlibrary{calc,shapes,shadows,decorations.pathmorphing}
\usepackage{mdframed}

%%% DEFINICIÓN DE COLORES
\definecolor{theoremBG}{RGB}{235,245,255}        % Fondo teorema más intenso
\definecolor{theoremBorder}{RGB}{210,225,250}    % Borde teorema más intenso
\definecolor{definitionBG}{RGB}{235,255,240}     % Fondo definición más intenso
\definecolor{definitionBorder}{RGB}{210,245,220} % Borde definición más intenso
\definecolor{remarkBG}{RGB}{255,240,230}         % Fondo observación más intenso
\definecolor{remarkBorder}{RGB}{250,220,205}     % Borde observación más intenso
\definecolor{easyBG}{RGB}{230,242,255}           % Fondo problema fácil más intenso
\definecolor{easyBorder}{RGB}{205,220,250}       % Borde problema fácil más intenso
\definecolor{mediumBG}{RGB}{255,240,225}         % Fondo problema medio más intenso
\definecolor{mediumBorder}{RGB}{250,220,205}     % Borde problema medio más intenso
\definecolor{hardBG}{RGB}{255,230,240}           % Fondo problema difícil más intenso
\definecolor{hardBorder}{RGB}{250,205,220}       % Borde problema difícil más intenso
\definecolor{exerciseBG}{RGB}{240,240,255}       % Fondo ejercicio más intenso
\definecolor{exerciseBorder}{RGB}{220,220,250}   % Borde ejercicio más intenso

%%% CONFIGURACIÓN DE ESTILOS TEOREMAS
\mdfdefinestyle{theoremstyle}{
    linecolor=theoremBorder,
    linewidth=0.4pt,                % Línea ligeramente más gruesa
    backgroundcolor=theoremBG!40,   % Fondo ligeramente más visible
    innertopmargin=10pt,
    innerbottommargin=10pt,
    innerleftmargin=10pt,
    innerrightmargin=10pt,
    skipabove=\topskip,
    skipbelow=\topskip,
}

% Aplicar el mismo patrón a los demás estilos
\mdfdefinestyle{definitionstyle}{
    linecolor=definitionBorder,
    linewidth=0.4pt,
    backgroundcolor=definitionBG!40,
    innertopmargin=10pt,
    innerbottommargin=10pt,
    innerleftmargin=10pt,
    innerrightmargin=10pt,
    skipabove=\topskip,
    skipbelow=\topskip,
}

\mdfdefinestyle{remarkstyle}{
    linecolor=remarkBorder,
    linewidth=0.4pt,
    backgroundcolor=remarkBG!40,
    innertopmargin=10pt,
    innerbottommargin=10pt,
    innerleftmargin=10pt,
    innerrightmargin=10pt,
    skipabove=\topskip,
    skipbelow=\topskip,
}

\mdfdefinestyle{easyproblemstyle}{
    linecolor=easyBorder,
    linewidth=0.4pt,
    backgroundcolor=easyBG!40,
    innertopmargin=10pt,
    innerbottommargin=10pt,
    innerleftmargin=10pt,
    innerrightmargin=10pt,
    skipabove=\topskip,
    skipbelow=\topskip,
}

\mdfdefinestyle{mediumproblemstyle}{
    linecolor=mediumBorder,
    linewidth=0.4pt,
    backgroundcolor=mediumBG!40,
    innertopmargin=10pt,
    innerbottommargin=10pt,
    innerleftmargin=10pt,
    innerrightmargin=10pt,
    skipabove=\topskip,
    skipbelow=\topskip,
}

\mdfdefinestyle{hardproblemstyle}{
    linecolor=hardBorder,
    linewidth=0.4pt,
    backgroundcolor=hardBG!40,
    innertopmargin=10pt,
    innerbottommargin=10pt,
    innerleftmargin=10pt,
    innerrightmargin=10pt,
    skipabove=\topskip,
    skipbelow=\topskip,
}

\mdfdefinestyle{exercisestyle}{
    linecolor=exerciseBorder,
    linewidth=0.4pt,
    backgroundcolor=exerciseBG!40,
    innertopmargin=10pt,
    innerbottommargin=10pt,
    innerleftmargin=10pt,
    innerrightmargin=10pt,
    skipabove=\topskip,
    skipbelow=\topskip,
}

%%% DEFINICIÓN DE TEOREMAS
\newtheorem{theorem}{Teorema}[section]
\newtheorem{corollary}{Corolario}[theorem]
\newtheorem{lemma}[theorem]{Lema}

\theoremstyle{definition}
\newtheorem{definition}{Definición}[section]

\theoremstyle{remark}
\newtheorem*{remark}{Observación}

\newtheorem{example}{Ejemplo}[section]
\newtheorem{easyproblem}{Problema}[section]
\newtheorem{mediumproblem}{Problema}[section]
\newtheorem{hardproblem}{Problema}[section]
\newtheorem{exercise}{Ejercicio}[section]

%%% APLICAR ESTILOS A LOS TEOREMAS
\surroundwithmdframed[style=theoremstyle]{theorem}
\surroundwithmdframed[style=theoremstyle]{corollary}
\surroundwithmdframed[style=theoremstyle]{lemma}
\surroundwithmdframed[style=definitionstyle]{definition}
\surroundwithmdframed[style=remarkstyle]{remark}
\surroundwithmdframed[style=theoremstyle]{example}
\surroundwithmdframed[style=easyproblemstyle]{easyproblem}
\surroundwithmdframed[style=mediumproblemstyle]{mediumproblem}
\surroundwithmdframed[style=hardproblemstyle]{hardproblem}
\surroundwithmdframed[style=exercisestyle]{exercise}

%%% ESTILO PARA DEMOSTRACIONES
\renewenvironment{proof}[1][\proofname]
    {\par\noindent\textit{#1.}\space}
    {\hfill$\square$\par\vspace{\baselineskip}}

%%% LENGUAJES DE PROGRAMACIÓN
\usepackage{verbatim}

%%% HIPERVÍNCULOS
\usepackage[
	breaklinks=true, 
	colorlinks=true, 
	linkcolor=blue,
	citecolor=blue, 
	urlcolor=gray
]{hyperref}
%% Divide los hipervínculos
\usepackage{xurl}

% Creative commons
\usepackage[scale=1.5]{ccicons}

%%% ESTILO DE REFERENCIAS
\usepackage{natbib}
\usepackage{apalike}

%%% ICONOS
\usepackage{marvosym}

%%% TEXTO GENERICO
\usepackage{lipsum}

%%% INDEX
\usepackage{imakeidx}
\makeindex[columns=3, title=Indice]

%Para paquetes .cvs
\usepackage{csvsimple}
\usepackage{xpatch}
\usepackage{etoolbox} % Para filtrar información en tablas.
\usepackage{booktabs} % Para hacer tablas bonitas desde un .csv

%Para escribir música
%\usepackage{musixtex}

%Estilo para programación.
\usepackage{listings}
\usepackage{color}
% http://paletton.com/#uid=7050t0kkJkJsntwoyp6gYgoddc4
%\usepackage[dvipsnames]{xcolor}
%	\definecolor{sec}{HTML}{DD5C14}
%	\definecolor{band}{HTML}{EE9C52}

% Paquete para pseudocódigo:
% \usepackage[Algoritmo]{algorithm}
\usepackage[spanish,onelanguage,ruled,vlined]{algorithm2e}
\usepackage[noend]{algpseudocode}

% Configuración para en ambiente de pseudocódigo:
\newcommand\mycommfont[1]{\footnotesize\ttfamily\textcolor{blue}{#1}}
\SetCommentSty{mycommfont}
\SetKwInput{KwInput}{Entrada}                
\SetKwInput{KwOutput}{Salida}             

% Ejemplo de pseudo código: 
\begin{comment}
\begin{algorithm}[H]
\DontPrintSemicolon
\BlankLine
\KwInput{Your Input}
\KwOutput{Your output}
\KwData{Testing set $x$}
$\sum_{i=1}^{\infty} := 0$ \tcp*{this is a comment}
\tcc{Now this is an if...else conditional loop}
\If{Condition 1}
{
Do something    \tcp*{this is another comment}
\If{sub-Condition}
{Do a lot}
}
\ElseIf{Condition 2}
{
Do Otherwise \;
\tcc{Now this is a for loop}
\For{sequence}    
{ 
loop instructions
}
}
\Else
{
Do the rest
}	
\tcc{Now this is a While loop}
\While{Condition}
{
Do something\;
}	
\caption{Example code}
\end{algorithm}
\end{comment}

% Definición de colores:
\definecolor{mygreen}{rgb}{0,0.6,0}
\definecolor{mygray}{rgb}{0.5,0.5,0.5}
\definecolor{mymauve}{rgb}{0.58,0,0.82}
\definecolor{orange}{RGB}{255,127,0}

% Estilo para escribir código de programación en el documento:
\renewcommand{\lstlistingname}{Código}
\lstset{ 
	keywordstyle=\color{blue},                % keyword style.	
	basicstyle=\footnotesize\ttfamily,        % size of fonts used for the code.
	commentstyle=\color{mygreen}\ttfamily,    % comment style.
	stringstyle=\color{mymauve},     % string literal style.
	backgroundcolor=\color{gray!5},          % choose the background color.
	rulecolor=\color{gray!5},                % border of the console.
	upquote=true,
	numbers=left, numberstyle=\tiny\color{gray}, stepnumber=1, numbersep=8pt, 
	% Formato de numeración.
	showstringspaces=false,
	breaklines=true,                 % automatic line breaking only at whitespace.
	frameround=ftff,
	frame=single,
	belowcaptionskip=-1.4em,
	belowskip=0em,
	aboveskip=-1.4em,
	captionpos=b,                    % sets the caption-position to bottom.
	escapeinside={\%*}{*)},          % if you want to add LaTeX within your code.
	literate={á}{{\'a}}1 {ã}{{\~a}}1 {é}{{\'e}}1 {í}{{\'i}}1 {ú}{{\'u}}1  {ó}{{\'o}}1,
}
% Ejemplo de  entorno de programación:
\begin{comment}
\begin{changemargin}{1.5cm}{1.5cm}
\begin{lstlisting}[language=python, caption={My caption},captionpos=b]
In [1]: logstart historia.py
\end{lstlisting}
\end{changemargin}
\end{comment}

% Gráficos, animaciones y diagramas.

\usepackage[all,color,dvips]{xy}
\usepackage{graphicx}
\DeclareGraphicsExtensions{.pdf,.png,.jpg}
%\usepackage{transparent}
\usepackage{eso-pic}



\usepackage{animate}
\usepackage {pst-solides3d}
\usepackage{pstricks}
\usepackage{pst-grad}
\usepackage{pst-3dplot}
\usepackage{pst-plot}
\usepackage{psfrag,pst-node} % Diagramas de Flujo
% \usepackage{pstcol} % Color para los Diagramas

% Paquetes para escribir expresiones matemáticas.
\usepackage{amsmath, amsthm, amsfonts, latexsym,cancel}
\usepackage{amsbsy}
\usepackage{thmtools}
\usepackage{amssymb}
\usepackage{xargs}
\usepackage{cancel} %Para tachar en formulas.
\usepackage{chemfig} % Fórmulas Químicas.
\usepackage[version=3]{mhchem}
\usepackage{epic,carom}
\usepackage{siunitx} % Para escribir unidades métricas. 

%Paquetes para tablas.
\usepackage{array} % Paquete para el diseño de tablas. 
\usepackage{multirow} % Para poder unir filas en las tablas
\usepackage{colortbl} % Para colorear tablas
\usepackage{longtable} % Controla el largo de la tabla
\usepackage{rotating} % Rotar tablas

% Maxima.
%\usepackage[amsmath]{maxiplot}

% Insertar etiquetas a las imágenes.
\usepackage{caption}[2013/02/03]
\usepackage{subcaption}
%Ejemplo de entorno de imagen con subcations:
%\begin{figure}
%\begin{minipage}[b]{.5\linewidth}
%  \centering\rule{2cm}{2cm}
% \subcaption{Primera subfigura}\label{fig:1a}
%\end{minipage}%
%\begin{minipage}[b]{.5\linewidth}
%  \centering\rule{2cm}{2cm$\BB$}
% \subcaption{Segunda subfigura}\label{fig:1b}
%\end{minipage}
%\caption{Figura con subfiguras}\label{fig:1}
%\end{figure}

% Tipo de letra del documento.
\usepackage{mathptmx}                 
\usepackage[scaled=.90]{helvet}         
\usepackage{courier}                
\usepackage{dsfont} 

% Numeración de las ecuaciones.
\usepackage{enumerate}
\numberwithin{equation}{section}

%Para partir enumeración.
\usepackage{enumitem}

% Espacio de líneas.
\linespread{1.2}
\usepackage{xspace} %Espacios después de los comandos.

% ARGUMENTOS VARIABLES
\usepackage{xargs}

% Lógica booleana para definir nuevos comandos:
\usepackage{ifthen}

% Example de comando usando el paquete ifthen:
\newcommand{\printTrueOrFalse}[1]
{
	\ifthenelse{\equal{#1}{true}}{TRUE}{}
	\ifthenelse{\equal{#1}{false}}{FALSE}{}
}

% NOTAS
\setlength {\marginparwidth }{2cm}
\usepackage[colorinlistoftodos, prependcaption]{todonotes}
\newcommandx{\note}[2][1=]{
	\todo[
	linecolor=green,
	backgroundcolor=green!25,
	bordercolor=green,#1]{#2}
}


% Definición de comandos matemáticos:
\newcommand{\KK}{\mathbb{K}} %Un cuerpo cualquiera.
\newcommand{\EE}{\mathcal{E}} %Espacio euclidiano.
\newcommand{\FF}{\mathbb{F}} %Campo comutativo.
\newcommand{\HH}{\mathbb{H}} %Semiespacio real superior.
\newcommand{\RR}{\mathbb{R}} %Números Reales.
\newcommand{\ZZ}{\mathbb{Z}} %Número Enteros.
\newcommand{\NN}{\mathbb{N}} %Numero Natureles.
\newcommand{\NNO}{\mathbb{N}_{_0}} %Numero Natureles.
\newcommand{\QQ}{\mathbb{Q}} %Números Racionales.
\newcommand{\CC}{\mathbb{C}} %Números Complejos
\newcommand{\RP}{\mathbb{RP}} %Plano proyectivo.
\newcommand{\A}{\mathcal{A}} %Atlas.
\newcommand{\B}{\mathcal{B}} %Atlas.
\newcommand{\BB}{\mathbb{B}} %Atlas.
\newcommand{\E}{\mathbb{E}} %Variedad Euclidiana.
\newcommand{\VV}{\mathbb{V}} %Espacio Vectorial V.
\newcommand{\hw}{\hspace{0.3cm}} %Espacio de 0.2cm.
\newcommand{\J}{\pmb{J}} %Jacobiano.
\newcommand{\ck}{\;_{\widetilde{C}^{k}}\,}  %Relación C^k - compatibles.
\newcommand{\cl}{\;_{\widetilde{C}^{l}}\,}  %Relación C^l - compatibles.
\newcommand{\cp}{\;_{\widetilde{p}}\,}  %Relación p - relacionados.
\newcommand{\ckr}{\;_{\widetilde{C}^{r}}\,} %Relacion C^r - compatibles.
\newcommand{\G}{\;_{\widetilde{G}}\;} 
\newcommand{\Ck}{_{\widetilde{C}^{k}}\,}
\newcommand{\C}{\mathcal{C}}
\newcommand{\cd}{\;|\;}
\newcommand{\fl}{f\hspace{0.0499em}l}
\newcommand{\fii}{f\hspace{0.03em}i}

% Comandos para que el csvsimple reconozca las tablas
\newcommand{\standardroman}[1]{\romannumeral\value{#1}}
\makeatletter
\xpatchcmd{\csv@breakline@kernel}{\roman}{\standardroman}{}{}
\xpatchcmd{\csv@current@col}{\roman}{\standardroman}{}{}
\xpatchcmd{\set@csv@autohead}{\roman}{\standardroman}{}{}
\xpatchcmd{\set@csv@head}{\roman}{\standardroman}{}{}
\xpatchcmd{\set@csv@nohead}{\roman}{\standardroman}{}{}
\makeatother


\def\SS{\mathbb{S}} %Esfera.
\def\TT{\mathbb{T}} %Toro.
\def\AA{\mathbb{A}} %Módulo A.
\def\LL{\mathcal{L}} %Familia de rectas. 
\def\PP{\mathcal{P}} %Familia de planos.


%Definición de operadores.
\DeclareMathOperator{\sen}{sen\,} %Función seno.
\DeclareMathOperator{\senh}{senh\,} %Función seno hiperbólico.
\DeclareMathOperator{\sech}{sech\,} %Función secante hiperbólico.
\DeclareMathOperator{\rot}{rot\,} %Rotacional.
\DeclareMathOperator{\dive}{div\,} %Divergencia.
\DeclareMathOperator{\nulo}{nul\,} %Nulo.
\DeclareMathOperator{\im}{img} %Imagen.
\DeclareMathOperator{\id}{id} %Función identidad.
\DeclareMathOperator{\Tr}{Tr\,} %Traza. 
\DeclareMathOperator{\Crit}{Crit\,} %Cunjunto de puntos críticos.
\DeclareMathOperator{\md}{\;mod} %Modulo.
\DeclareMathOperator{\ord}{ord} %Orden.
\DeclareMathOperator{\mcd}{mcd} %Máximo común divisor.
\DeclareMathOperator{\mcm}{mcm} %Mínimo común múltiplo
\DeclareMathOperator{\Gr}{Gr} %Gráfica.
\DeclareMathOperator{\orb}{orb} %Orbita.
\DeclareMathOperator{\dif}{dif} %Conjunto de difeomorfismos.
\DeclareMathOperator{\GL}{GL} %Grupo Especial Lineal 
\DeclareMathOperator{\Int}{Int} %Interior de un grupo 
\DeclareMathOperator{\sgn}{sgn} %Signo de una permutación
\DeclareMathOperator{\sop}{sop} %Soporte de una partición
\DeclareMathOperator{\dom}{dom} %Domino
\DeclareMathOperator{\ran}{ran} %Rango
\DeclareMathOperator{\Arg}{Arg} %Argumento de un número complejo
\DeclareMathOperator{\End}{End} %Endomorfismos
\DeclareMathOperator{\vol}{vol} %Volumen
\DeclareMathOperator{\Hom}{Hom} %Homomorfismos
\DeclareMathOperator{\Der}{Der} %Derivaciones
\DeclareMathOperator{\SO}{SO} %Grupo ortogonal especial
\DeclareMathOperator{\U}{U} %Grupo unitario
\DeclareMathOperator{\SU}{SU} %Grupo unitario especial
\DeclareMathOperator{\SL}{SL} %Grupo lineal especial
%\DeclareMathOperator{\tr}{tr} %Traza de una matriz
\DeclareMathOperator{\diag}{diag} %Matriz diagonal
\DeclareMathOperator{\Lie}{Lie} %Matriz diagonal
\DeclareMathOperator{\proy}{proy} %Matriz diagonal
\DeclareMathOperator{\card}{card} %Cardinal de un conjunto
\DeclareMathOperator{\instab}{instab} %Inestabilidad de un AC.
\DeclareMathOperator{\gen}{gen} %Generador.
\DeclareMathOperator{\Colu}{Col} %Columna.
\DeclareMathOperator{\vari}{var} %Varianza.
\DeclareMathOperator{\adj}{adj} %Varianza.
\DeclareMathOperator*{\argMin}{arg\, min} % Argumento minimal.
\DeclareMathOperator{\Jac}{\pmb{J}} % Jacobiano


% LISTA DE COMANDOS GENERALIZADOS:
% Comando nuevo para calcelar con diferentes colores.
\newcommand{\Cancel}[2][black]{\renewcommand\CancelColor{\color{#1}}\cancel{#2}}

% Comando para una lista de puntos. 
\newcommandx{\points}[3][1=0, 3=n]{#2_{_{#1}},\ldots,#2_{_{#3}}} 

% COMANDOS MATEMÁTICOS SIMPLES:

% Función del argumento mínimo:
\newcommand*\argmin[1]{\argMin_{#1}} 

% Insertar fondo de página en el documento:
\newcommandx{\BackgroundPic}[1]{
	\put(-3,0){
		\parbox[b][\paperheight]{\paperwidth}{%
			\vfill
			\centering
			{\includegraphics[width=\paperwidth,height=\paperheight]{#1}}%
			%{\transparent{#2}\includegraphics[width=\paperwidth,height=\paperheight]{#1}}% Está linea solo funciona con pfdtex
			\vfill
}}}

% DEFINICIÓN DE NUEVOS ACENTOS:

% Acento de arco:
\DeclareFontFamily{OMX}{yhex}{}
\DeclareFontShape{OMX}{yhex}{m}{n}{<->yhcmex10}{}
\DeclareSymbolFont{yhlargesymbols}{OMX}{yhex}{m}{n}
\DeclareMathAccent{\arco}{\mathord}{yhlargesymbols}{"F3}


% Acentos:
\begin{comment}
    $\arco{AB}$ $\widehat{ABC}$
\end{comment}

% Comandos para notas:
\usepackage[colorinlistoftodos,prependcaption]{todonotes}
\newcommandx{\unsure}[2][1=]{\todo[linecolor=red,backgroundcolor=red!25,bordercolor=red,#1]{#2}}
\newcommandx{\change}[2][1=]{\todo[linecolor=blue,backgroundcolor=blue!25,bordercolor=blue,#1]{#2}}
\newcommandx{\info}[2][1=]{\todo[linecolor=green,backgroundcolor=green!25,bordercolor=green,#1]{#2}}
\newcommandx{\improvement}[2][1=]{\todo[linecolor=Plum,backgroundcolor=Plum!25,bordercolor=Plum,#1]{#2}}
\newcommandx{\thiswillnotshow}[2][1=]{\todo[disable,#1]{#2}}

% Renombrado comandos: 

\renewcommand\figureautorefname{Figura}
\renewcommand\pageautorefname{pagína}
\renewcommand\algorithmautorefname{Algoritmo}
\renewcommand\theoremautorefname{Teorema}


% Comando para hyperefereces:
\newcommand{\sref}[2]{\hyperref[#2]{#1 \ref*{#2}}}

%Paquetes para el diseño del encabezado.
\usepackage{fancyhdr}
\usepackage{fancyvrb}
%\usepackage{parskip} % Quita la indentación inicial de los parráfos. 

\pagestyle{fancy}
\renewcommand{\sectionmark}[1]{\markboth{#1}{}} % Quita la numeración de los capitulos en el encabezado.
\fancyhf{}
%\fancyhead[LE]{\nouppercase{\small\textit \thepage}\small\it{/Tensor Geométrico}} %Ubica el título del libro en pares.
\fancyhead[RO]{\nouppercase{\small\textit\leftmark \,/}{\small\textit \thepage}} %Ubica el título del capitulo en impares.
\fancypagestyle{plain}{\fancyhead{}}
\renewcommand{\headrulewidth}{0pt}

%Corrige la separación de palabras.
%\usepackage[none]{hyphenat} 
\hyphenation{corres-pondencia equiva-lente geo-metría posibi-lidad axio-mas}

%Líneas huerfanas.
\tolerance 1414
\hbadness 1414
%\emergencystretch 1.5em
\hfuzz 0.3pt
\vfuzz \hfuzz
%\raggedbottom
%Lo anterior está en prueba
\clubpenalty=1000
\widowpenalty=1000

\def\changemargin#1#2{\list{}{\rightmargin#2\leftmargin#1}\item[]}
\let\endchangemargin=\endlist

\setlength{\parindent}{0cm}
% Define el espacio entre parrafos. 
\setlength{\parskip}{2mm}

% Espacio entre teorema y demostración:
\usepackage{etoolbox}

% Árboles
\usepackage{qtree}

% \definecolor{micolor}{RGB}{255,255,240}
% \pagecolor{micolor}

