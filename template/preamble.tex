% preamble.tex - Contains all packages and configurations
\documentclass[12pt, a4paper, oneside]{book}
%%% BASIC PACKAGES
\usepackage[T1]{fontenc}
\usepackage[utf8]{inputenc}
\usepackage{lmodern}
\usepackage[spanish, es-noshorthands]{babel}
\usepackage{comment}
\usepackage{textcomp}

%%% DOCUMENT GEOMETRY
\usepackage[top=2.5cm, bottom=2.5cm, right=2.5cm, left=2.5cm, headheight=15pt, marginparwidth=2.5cm]{geometry}
\usepackage{multicol}
\usepackage{lscape}

%%% MATH PACKAGES
\usepackage{amsmath, amsthm, amsfonts, latexsym, cancel}
\usepackage{amsbsy}
\usepackage{thmtools}
\usepackage{amssymb}
\usepackage{mathtools}

%%% STYLE AND DESIGN PACKAGES
\usepackage[dvipsnames]{xcolor}
\usepackage{tikz}
\usetikzlibrary{calc,shapes,shadows,decorations.pathmorphing,matrix,positioning}
\usepackage{tikz-qtree}
\usepackage{mdframed}

%%% CODE LISTINGS
\usepackage{listings}
\usepackage{tcolorbox}
\tcbuselibrary{listings, skins, breakable}
\renewcommand{\lstlistingname}{Código}

% Modern code color palette
\definecolor{codeBackground}{RGB}{250, 250, 252}
\definecolor{codeFrame}{RGB}{218, 220, 235}
\definecolor{codeAccent}{RGB}{99, 102, 241}
\definecolor{codeKeyword}{RGB}{124, 58, 237}
\definecolor{codeString}{RGB}{5, 150, 105}
\definecolor{codeComment}{RGB}{148, 163, 184}
\definecolor{codeNumber}{RGB}{180, 190, 205}
\definecolor{codeFunction}{RGB}{37, 99, 235}
\definecolor{codeVariable}{RGB}{217, 70, 239}

\lstset{
    basicstyle=\ttfamily\footnotesize\linespread{1.1}\selectfont,
    keywordstyle=\color{codeKeyword}\bfseries,
    keywordstyle=[2]\color{codeFunction},
    commentstyle=\color{codeComment}\itshape,
    stringstyle=\color{codeString},
    numbers=left,
    numberstyle=\tiny\color{codeNumber}\ttfamily,
    numbersep=10pt,
    breaklines=true,
    breakatwhitespace=true,
    showstringspaces=false,
    tabsize=4,
    extendedchars=true,
    inputencoding=utf8,
    literate={á}{{\'a}}1 {é}{{\'e}}1 {í}{{\'i}}1 {ó}{{\'o}}1 {ú}{{\'u}}1
             {ñ}{{\~n}}1 {Á}{{\'A}}1 {É}{{\'E}}1 {Í}{{\'I}}1 {Ó}{{\'O}}1
             {Ú}{{\'U}}1 {Ñ}{{\~N}}1
}

% Custom tcolorbox style for code listings
\newtcblisting{codebox}[2][]{
    listing only,
    listing options={language=#2, style=tcblatex},
    enhanced,
    breakable,
    colback=codeBackground,
    colframe=codeFrame,
    boxrule=0.4pt,
    left=8pt,
    right=8pt,
    top=10pt,
    bottom=10pt,
    arc=2pt,
    borderline west={2pt}{0pt}{codeAccent!70},
    #1
}

% Style for lstinputlisting with tcolorbox
\tcbset{
    codestyle/.style={
        listing only,
        enhanced,
        breakable,
        colback=codeBackground,
        colframe=codeFrame,
        boxrule=0.4pt,
        left=16pt,
        right=8pt,
        top=8pt,
        bottom=10pt,
        arc=2pt,
        borderline west={2pt}{0pt}{codeAccent!70},
        fonttitle=\bfseries\small\color{black!80},
        coltitle=black!80,
        colbacktitle=codeFrame!50,
        toptitle=3pt,
        bottomtitle=3pt,
        attach boxed title to top left={yshift=-2mm, xshift=2mm},
        boxed title style={colback=codeFrame!50, colframe=codeFrame, boxrule=0pt, arc=1pt},
    }
}

\newtcbinputlisting{\codefile}[2][]{
    codestyle,
    listing file={#2},
    #1
}

%%% BOOK-SPECIFIC PACKAGES
\usepackage{fancyhdr}
\usepackage{booktabs}
\usepackage{minitoc}
\usepackage{epigraph}
\usepackage{titlesec}

%%% ALGORITHMS
\usepackage[ruled,vlined,spanish,onelanguage]{algorithm2e}
\SetKwInput{KwEntrada}{Entrada}
\SetKwInput{KwSalida}{Salida}
\SetKwFor{ParaCada}{para cada}{hacer}{fin}
\SetKwIF{Si}{SinoSi}{Sino}{si}{entonces}{sino si}{sino}{fin}
\SetKwFor{Para}{para}{hacer}{fin}
\SetKw{Retornar}{retornar}

%%% BIBLIOGRAPHY AND REFERENCES
\usepackage[backend=biber, style=apa, sorting=ynt]{biblatex}
\usepackage{csquotes}

%%% INDEX AND GLOSSARY
\usepackage{imakeidx}
\usepackage[toc]{glossaries}
\makeindex[columns=3, title=Índice Alfabético]

%%% GRAPHICS AND FIGURES
\usepackage{graphicx}
\usepackage{caption}
\usepackage{subcaption}
\graphicspath{{./images/}}

%%% HYPERLINKS
\usepackage[
    breaklinks=true,
    colorlinks=true,
    linkcolor=blue,
    citecolor=blue,
    urlcolor=gray
]{hyperref}
\usepackage{xurl}

%%% COLOR DEFINITIONS
\definecolor{theoremBG}{RGB}{235,245,255}
\definecolor{theoremBorder}{RGB}{210,225,250}
\definecolor{definitionBG}{RGB}{235,255,240}
\definecolor{definitionBorder}{RGB}{210,245,220}
\definecolor{remarkBG}{RGB}{255,240,230}
\definecolor{remarkBorder}{RGB}{250,220,205}
\definecolor{easyBG}{RGB}{230,242,255}
\definecolor{easyBorder}{RGB}{205,220,250}
\definecolor{mediumBG}{RGB}{255,240,225}
\definecolor{mediumBorder}{RGB}{250,220,205}
\definecolor{hardBG}{RGB}{255,230,240}
\definecolor{hardBorder}{RGB}{250,205,220}
\definecolor{exerciseBG}{RGB}{240,240,255}
\definecolor{exerciseBorder}{RGB}{220,220,250}

%%% THEOREM STYLES
\mdfdefinestyle{theoremstyle}{
    linecolor=theoremBorder,
    linewidth=0.4pt,
    backgroundcolor=theoremBG!40,
    innertopmargin=10pt,
    innerbottommargin=10pt,
    innerleftmargin=10pt,
    innerrightmargin=10pt,
    skipabove=\topskip,
    skipbelow=\topskip,
}

\mdfdefinestyle{definitionstyle}{
    linecolor=definitionBorder,
    linewidth=0.4pt,
    backgroundcolor=definitionBG!40,
    innertopmargin=10pt,
    innerbottommargin=10pt,
    innerleftmargin=10pt,
    innerrightmargin=10pt,
    skipabove=\topskip,
    skipbelow=\topskip,
}

\mdfdefinestyle{remarkstyle}{
    linecolor=remarkBorder,
    linewidth=0.4pt,
    backgroundcolor=remarkBG!40,
    innertopmargin=10pt,
    innerbottommargin=10pt,
    innerleftmargin=10pt,
    innerrightmargin=10pt,
    skipabove=\topskip,
    skipbelow=\topskip,
}

\mdfdefinestyle{easyproblemstyle}{
    linecolor=easyBorder,
    linewidth=0.4pt,
    backgroundcolor=easyBG!40,
    innertopmargin=10pt,
    innerbottommargin=10pt,
    innerleftmargin=10pt,
    innerrightmargin=10pt,
    skipabove=\topskip,
    skipbelow=\topskip,
}

\mdfdefinestyle{mediumproblemstyle}{
    linecolor=mediumBorder,
    linewidth=0.4pt,
    backgroundcolor=mediumBG!40,
    innertopmargin=10pt,
    innerbottommargin=10pt,
    innerleftmargin=10pt,
    innerrightmargin=10pt,
    skipabove=\topskip,
    skipbelow=\topskip,
}

\mdfdefinestyle{hardproblemstyle}{
    linecolor=hardBorder,
    linewidth=0.4pt,
    backgroundcolor=hardBG!40,
    innertopmargin=10pt,
    innerbottommargin=10pt,
    innerleftmargin=10pt,
    innerrightmargin=10pt,
    skipabove=\topskip,
    skipbelow=\topskip,
}

\mdfdefinestyle{exercisestyle}{
    linecolor=exerciseBorder,
    linewidth=0.4pt,
    backgroundcolor=exerciseBG!40,
    innertopmargin=10pt,
    innerbottommargin=10pt,
    innerleftmargin=10pt,
    innerrightmargin=10pt,
    skipabove=\topskip,
    skipbelow=\topskip,
}

%%% THEOREM DEFINITIONS
\newtheorem{theorem}{Teorema}[chapter]
\newtheorem{corollary}{Corolario}[theorem]
\newtheorem{lemma}[theorem]{Lema}

\theoremstyle{definition}
\newtheorem{definition}{Definición}[chapter]

\theoremstyle{remark}
\newtheorem*{remark}{Observación}

\newtheorem{example}{Ejemplo}[chapter]
\newtheorem{easyproblem}{Problema}[chapter]
\newtheorem{mediumproblem}{Problema}[chapter]
\newtheorem{hardproblem}{Problema}[chapter]
\newtheorem{exercise}{Ejercicio}[chapter]

%%% APPLY STYLES TO THEOREMS
\surroundwithmdframed[style=theoremstyle]{theorem}
\surroundwithmdframed[style=theoremstyle]{corollary}
\surroundwithmdframed[style=theoremstyle]{lemma}
\surroundwithmdframed[style=definitionstyle]{definition}
\surroundwithmdframed[style=remarkstyle]{remark}
\surroundwithmdframed[style=theoremstyle]{example}
\surroundwithmdframed[style=easyproblemstyle]{easyproblem}
\surroundwithmdframed[style=mediumproblemstyle]{mediumproblem}
\surroundwithmdframed[style=hardproblemstyle]{hardproblem}
\surroundwithmdframed[style=exercisestyle]{exercise}

%%% PROOF STYLE
\renewenvironment{proof}[1][\proofname]
{\par\noindent\textit{#1.}\space}
{\hfill$\square$\par\vspace{\baselineskip}}

%%% CUSTOM CHAPTER FORMATTING
\titleformat{\chapter}[display]
{\normalfont\huge\bfseries}
{\chaptertitlename\ \thechapter}
{20pt}
{\Huge}
\titlespacing*{\chapter}{0pt}{50pt}{40pt}

%%% CUSTOM SECTION FORMATTING
\titleformat{\section}
{\normalfont\Large\bfseries}
{\thesection}
{1em}
{}
\titlespacing*{\section}{0pt}{3.5ex plus 1ex minus .2ex}{2.3ex plus .2ex}

%%% HEADER AND FOOTER SETTINGS
\pagestyle{fancy}
\fancyhf{}
\renewcommand{\chaptermark}[1]{\markboth{\chaptername\ \thechapter.\ #1}{}}
\renewcommand{\sectionmark}[1]{\markright{\thesection.\ #1}}
\fancyhead[LE,RO]{\thepage}
\fancyhead[LO]{\textit{\rightmark}}
\fancyhead[RE]{\textit{\leftmark}}
\renewcommand{\headrulewidth}{0.4pt}
\renewcommand{\footrulewidth}{0pt}

%%% MATH OPERATORS
\DeclareMathOperator{\sen}{sen\,}
\DeclareMathOperator{\senh}{senh\,}
\DeclareMathOperator{\sech}{sech\,}
\DeclareMathOperator{\rot}{rot\,}
\DeclareMathOperator{\dive}{div\,}
\DeclareMathOperator{\nulo}{nul\,}
\DeclareMathOperator{\im}{img}
\DeclareMathOperator{\id}{id}
\DeclareMathOperator{\Tr}{Tr\,}
\DeclareMathOperator{\Crit}{Crit\,}
\DeclareMathOperator{\md}{\;mod}
\DeclareMathOperator{\ord}{ord}
\DeclareMathOperator{\mcd}{mcd}
\DeclareMathOperator{\mcm}{mcm}

%%% MATH SYMBOLS AND COMMANDS
\newcommand{\KK}{\mathbb{K}}
\newcommand{\EE}{\mathcal{E}}
\newcommand{\FF}{\mathbb{F}}
\newcommand{\HH}{\mathbb{H}}
\newcommand{\RR}{\mathbb{R}}
\newcommand{\ZZ}{\mathbb{Z}}
\newcommand{\NN}{\mathbb{N}}
\newcommand{\NNO}{\mathbb{N}_{_0}}
\newcommand{\QQ}{\mathbb{Q}}
\newcommand{\CC}{\mathbb{C}}

%%% PARAGRAPH SETTINGS
\setlength{\parindent}{0cm}
\setlength{\parskip}{2mm}

%%% ORPHANS AND WIDOWS PREVENTION
\clubpenalty=1000
\widowpenalty=1000
\displaywidowpenalty=1000

% --- Custom commands for title page ---
\makeatletter
\newcommand{\printbooktitle}{\@title}
\newcommand{\printbookauthor}{\@author}
\newcommand{\printbookdate}{\@date}
\makeatother
\let\cleardoublepage\clearpage
\addbibresource{bibliography/references.bib}
