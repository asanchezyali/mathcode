\chapter{La genesis del análisis de Fourier}

\epigraph{Respecto a las investigaciones de d'Alembert y Euler, ¿no podría añadirse que, si conocían esta expansión, hicieron un uso muy imperfecto de ella? Ambos estaban persuadidos de que una función arbitraria y discontinua nunca podría resolverse en series de este tipo, y ni siquiera parece que alguien hubiera desarrollado una constante en cosenos de arcos múltiples, el primer problema que tuve que resolver en la teoría del calor.}{\textit{J. Fourier}, 1808-9}

\section{Introducción histórica}

Durante el siglo XVIII, matemaáticos como Jean Le Rond d'Alembert(1717-1783) y Leonhard Euler (1707-1783) investigaron
la ecuación de onda para describir el movimiento de una cuerda vibrante \parencite{stein2003}. Aunque en 1753 Daniel
Bernulli (1700-1782) propuso que cualquier movimiento en la cuerda podía expresarse como una suma infinita de
oscilaciones sinusoidales (armónicos), Euler y d'Alembert Euler y d'Alembert rechazaron la propuesta de Bernoulli,
argumentando que una suma de funciones suaves y periódicas (como senos y cosenos) jamás podría representar una curva
que no fuera inherentemente suave o que tuviera comportamientos no periódicos en su origen
\parencite{rodriguez-zuazua}. En esa época, se creía firmemente que una función con discontinuidades nunca podría ser
representada por una serie de funciones continuas y suaves como el seno y el coseno.

La revolución en el pensamiento matemático llegó con Joseph Fourier (1768-1830) a principios del siglo XIX
\parencite{sensenbaugh2023}, \parencite{wiki-fourier}. Mientras servía como prefecto en Isère, Francia, Fourier comenzó
a experimentar con la difusión térmica, adoptando un enfoque fenomenológico: no buscacaba entender qué era el calor,
sino las leyes matemáticas que gobernaban su propagación. En su memoría de 1807, \textit{Mémoire sur la propagation de
    la chaleur dans les corps solides}, \parencite{fourier1807}, Fourier derivó la ecuación del calor basándose en el
principio de conservación de la energía y en la ley que establece que el flujo de calor es proporcional al gradiente
negativo de la temperatura.