% chapter1.tex - Lógica Proposicional

\chapter{Lógica Proposicional}

\section{Introducción}

La lógica matemática es el estudio de los métodos de razonamiento. En este capítulo introducimos los conceptos fundamentales de la lógica proposicional, que constituye la base del pensamiento matemático riguroso.

\begin{definition}
Una \textbf{proposición} es un enunciado declarativo que puede ser verdadero o falso, pero no ambos simultáneamente.
\end{definition}

\begin{example}
Los siguientes son ejemplos de proposiciones:
\begin{itemize}
    \item ``2 + 2 = 4'' (verdadera)
    \item ``La Tierra es plana'' (falsa)
    \item ``Existen infinitos números primos'' (verdadera)
\end{itemize}
\end{example}

\section{Conectivos Lógicos}

Los conectivos lógicos permiten construir proposiciones compuestas a partir de proposiciones simples.

\begin{definition}
Los \textbf{conectivos lógicos} básicos son:
\begin{itemize}
    \item Negación ($\neg$): ``no''
    \item Conjunción ($\land$): ``y''
    \item Disyunción ($\lor$): ``o''
    \item Implicación ($\rightarrow$): ``si... entonces''
    \item Bicondicional ($\leftrightarrow$): ``si y solo si''
\end{itemize}
\end{definition}

\section{Tablas de Verdad}

Las tablas de verdad determinan el valor de verdad de proposiciones compuestas.

\begin{theorem}
Para toda proposición $p$, se cumple que $\neg(\neg p) \equiv p$.
\end{theorem}

\begin{proof}
Construimos la tabla de verdad:
\begin{center}
\begin{tabular}{|c|c|c|}
\hline
$p$ & $\neg p$ & $\neg(\neg p)$ \\
\hline
V & F & V \\
F & V & F \\
\hline
\end{tabular}
\end{center}
Observamos que las columnas $p$ y $\neg(\neg p)$ son idénticas.
\end{proof}

\section{Ejercicios}

\begin{exercise}
Construye la tabla de verdad para la proposición $(p \land q) \rightarrow p$.
\end{exercise}

\begin{exercise}
Demuestra que $(p \rightarrow q) \equiv (\neg p \lor q)$.
\end{exercise}
