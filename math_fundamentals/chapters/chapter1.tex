\chapter{Los números reales y sus propiedades}

En esta sección introducimos el sistema de los números reales y presentamos sus propiedades fundamentales. Como es bien
conocido, los números reales forma la base para el análisis matemático. Desde los griegos ya se sabía que los números
racionales, es decir, cocientes de números enteros, no eran suficientes para describir todas las cantidades
geométricas. Por ejemplo, en un triángulo reactángulo con catetos de longitud $1$, por el teorema de Pitágoras, la
longitud $c$ de la hipotenusa es tal que $c^2 = 2$, es fácil probar que no existe ningún número racional $p/q$ (con $p,
    q \in \ZZ$ y $q \neq 0$) que satisfaga esta ecuación. En efecto, si $c = p/q$ fuera una solución racional y tal que $p$
y $q$ no tuvieran factores primos en común, entonces $p^2 = 2q^2$ implica que $p^2$ es par, y por lo tanto $p$ es par.
Si $p$ es par, entonces existe un entero $k$ tal que $p = 2k$, y por lo tanto $p^2 = 4k^2$. Sustituyendo en la ecuación
original obtenemos $4k^2 = 2q^2$, o lo que es lo mismo, $q^2 = 2k^2$. Esto implica que $q^2$ es par, y por lo tanto $q$
es par. Pero si $p$ y $q$ son ambos pares, entonces tienen al menos el factor primo $2$ en común, lo cual contradice
nuestra suposición inicial. Por lo tanto, no existe ningún número racional que satisfaga la ecuación $c^2 = 2$. Sin
embargo el número la ecuación $c^2 = 2$ tiene una solución en los números reales, y a este número se denota por
$\sqrt{2}$. Para resolver este problema fue necesario ampliar el sistema de los número racionales a un sistema más
grande, el sistema de los números reales.

\begin{exercise}[Aproximación racional de $\sqrt{2}$]
    Como vimos, $\sqrt{2}$ no es un número racional. Sin embargo, podemos aproximarlo mediante fracciones. El método babilónico (o de Herón) genera una sucesión de aproximaciones racionales: dado $x_0 > 0$, se define
    \begin{equation*}
        x_{n+1} = \frac{1}{2}\left(x_n + \frac{2}{x_n}\right).
    \end{equation*}
    Implemente en Python una función que, dado un valor inicial $x_0$ y un número de iteraciones $n$, retorne la aproximación $x_n$ de $\sqrt{2}$. Utilice la biblioteca \texttt{fractions} para trabajar con números racionales exactos y observe cómo las fracciones se vuelven cada vez más complejas mientras se acercan a $\sqrt{2}$.
\end{exercise}

\section{Axiomas de la suma en $\RR$}
Comenzaremos por postular que existe un conjunto $\RR$, cuyos elementos llamaremos \textbf{números reales}, junto con
las operaciones de suma y multiplicación y una relación de orden.

\begin{definition}[Suma en $\RR$]
    La \textbf{suma} en los números reales es una función
    \begin{equation*}
        + : \RR \times \RR \longrightarrow \RR,
    \end{equation*}
    que asigna a cada par $(a, b)$ el elemento $+(a, b)$, denotado $a + b$ (notación infija). Esta operación satisface los siguientes axiomas:
    \begin{enumerate}
        \item[AS1.] \textbf{Conmutatividad:} Para todo $a, b \in \RR$, se cumple que $a + b = b + a$.
        \item[AS2.] \textbf{Asociatividad:} Para todo $a, b, c \in \RR$, se cumple que $(a + b) + c = a + (b + c)$.
        \item[AS3.] \textbf{Elemento neutro:} Existe un elemento $0 \in \RR$ tal que para todo $a \in \RR$, se cumple que $a + 0 =
                  a$.
        \item[AS4.] \textbf{Elemento inverso:} Para todo $a \in \RR$, existe un elemento $-a \in \RR$ tal que $a + (-a) = 0$.
    \end{enumerate}
\end{definition}

Con estos cuatro axiomas, el conjunto $\RR$ junto con la operación de suma es un \textbf{grupo abeliano}. Además, a
partir de estos axiomas se pueden demostrar las siguientes propiedades adicionales de la suma en $\RR$.
\begin{theorem}
    Para todo $a, b, c \in \RR$, se cumplen las siguientes propiedades:
    \begin{enumerate}
        \item $a + (-a) = 0$.
        \item $-(-a) = a$.
        \item Si $a + b = a + c$, entonces $b = c$.
        \item $0 + a = a$.
        \item Si para algún $a \in \RR$ se cumple que $a + b = a$, entonces $b = 0$.
    \end{enumerate}
\end{theorem}

\section{Axiomas de la multiplicación en $\RR$}
Ahora definimos la operación de multiplicación en los números reales mediante los siguientes axiomas.

\begin{definition}[Multiplicación en $\RR$]
    La \textbf{multiplicación} en los números reales es una función
    \begin{equation*}
        \cdot : \RR \times \RR \longrightarrow \RR,
    \end{equation*}
    que asigna a cada par $(a, b)$ el elemento $\cdot(a, b)$, denotado $a \cdot b$ o simplemente $ab$ (notación infija). Esta operación satisface los siguientes axiomas:
    \begin{enumerate}
        \item[AM1.] \textbf{Conmutatividad:} Para todo $a, b \in \RR$, se cumple que $a \cdot b = b \cdot a$.
        \item[AM2.] \textbf{Asociatividad:} Para todo $a, b, c \in \RR$, se cumple que $(a \cdot b) \cdot c = a \cdot (b \cdot c)$.
        \item[AM3.] \textbf{Elemento neutro:} Existe un elemento $1 \in \RR \setminus \{0\}$ tal que para todo $a \in \RR$, se cumple
              que $1 \cdot a = a$.
        \item[AM4.] \textbf{Elemento inverso:} Para todo $a \in \RR \setminus \{0\}$, existe un elemento $a^{-1} \in \RR \setminus
                  \{0\}$ tal que $a \cdot a^{-1} = 1$.
        \item[AM5.] \textbf{Distributividad:} Para todo $a, b, c \in \RR$, se cumple que $a \cdot (b + c) = a \cdot b + a \cdot c$.
    \end{enumerate}
\end{definition}

Se puede observar que los axiomas AM1 a AM4 son similares a los axiomas AS1 a AS4, y juntos implican que el conjunto
$\RR \setminus \{0\}$ junto con la operación de multiplicación es también un grupo abeliano. El axioma AM5 expresa una
relación entre las operaciones de suma y multiplicación.

\begin{theorem}
    Para todo $a, b, c \in \RR$, se cumplen las siguientes propiedades:
    \begin{enumerate}
        \item $a \cdot 0 = 0$.
        \item Si $a \cdot b = 0$, entonces $a = 0$ o $b = 0$.
        \item Si $a \cdot b = a \cdot c$ y $a \neq 0$, entonces $b = c$.
        \item $a \cdot 1 = a$.
        \item Si para algún $a \in \RR \setminus \{0\}$ se cumple que $a \cdot b = a$, entonces $b = 1$.
    \end{enumerate}
\end{theorem}

El conjunto de los números reales junto con las operaciones de suma y multiplicación que hemos definido hasta ahora se
denomina un \textbf{cuerpo} y es una estructura algebraica fundamental en matemáticas que usualmente se denota como
$(\RR, +, \cdot)$. En general, si un conjunto $K$ junto con dos operaciones binarias $+$ y $\cdot$ satisface los
axiomas AS1 a AS4 y AM1 a AM5, entonces decimos que $(K, +, \cdot)$ es un cuerpo.

\begin{exercise}
    Considere un sistema con dos elementos $\alpha$ y $\beta$ y las siguientes reglas de suma y multiplicación:
    \begin{align*}
        \alpha + \alpha     & = \alpha, & \alpha + \beta     & = \beta,  & \beta + \alpha     & = \beta,  & \beta + \beta     & = \alpha, \\
        \alpha \cdot \alpha & = \alpha, & \alpha \cdot \beta & = \alpha, & \beta \cdot \alpha & = \alpha, & \beta \cdot \beta & = \beta.
    \end{align*}
    Demuestre que este sistema forma un cuerpo.
\end{exercise}

\begin{exercise}
    Considere los números de la forma $a + b\sqrt{6}$ donde $a$ y $b$ son racionales. ¿Satisface este conjunto los axiomas de un cuerpo?
\end{exercise}

\begin{exercise}[Verificación de axiomas de cuerpo]
    Implemente en Python una clase \texttt{TwoElementField} que represente el cuerpo con dos elementos $\{\alpha, \beta\}$ del Ejercicio 1.2. La clase debe:
    \begin{enumerate}
        \item Definir las operaciones de suma y multiplicación según las tablas dadas.
        \item Incluir un método que verifique automáticamente cada uno de los axiomas AS1--AS4 y AM1--AM5 para todos los elementos
              del cuerpo.
        \item Identificar cuál elemento actúa como el $0$ (neutro aditivo) y cuál como el $1$ (neutro multiplicativo).
    \end{enumerate}
\end{exercise}

\section{Cuerpos finitos y sobrecarga de operadores}

Los números reales no son el único ejemplo de cuerpo. Un caso particularmente interesante son los \textbf{cuerpos
    finitos}, que tienen un número finito de elementos. Podemos construir cuerpos finitos utilizando la \textbf{aritmética
    modular}.

\begin{definition}[Aritmética modular]
    Dado un entero positivo $n$, decimos que dos enteros $a$ y $b$ son \textbf{congruentes módulo $n$}, y escribimos $a \equiv b \pmod{n}$, si y solo si $n$ divide a $a - b$. Equivalentemente, $a$ y $b$ tienen el mismo residuo al dividirse entre $n$. El conjunto de clases de equivalencia se denota
    \begin{equation*}
        \ZZ/n\ZZ = \{[0], [1], [2], \ldots, [n-1]\},
    \end{equation*}
    donde $[a]$ representa la clase de todos los enteros congruentes con $a$ módulo $n$. Las operaciones en $\ZZ/n\ZZ$ se definen como:
    \begin{align*}
        [a] + [b]     & = [a + b],     \\
        [a] \cdot [b] & = [a \cdot b].
    \end{align*}
\end{definition}

La aritmética modular puede visualizarse como un reloj: los números «ciclan» al llegar al módulo. Si el módulo es $12$
(como en un reloj), entonces $10 + 5 \equiv 3 \pmod{12}$, pues al avanzar $5$ horas desde las $10$, llegamos a las $3$.

Dado un número primo $p$, el conjunto $\ZZ/p\ZZ$ con las operaciones de suma y multiplicación módulo $p$ forma un
cuerpo. Por ejemplo, en $\ZZ/5\ZZ$ tenemos que $3 + 4 \equiv 2 \pmod{5}$ (pues $7 = 5 + 2$) y $3 \cdot 4 \equiv 2
    \pmod{5}$ (pues $12 = 2 \cdot 5 + 2$).

\begin{theorem}
    Si $p$ es un número primo, entonces $(\ZZ/p\ZZ, +, \cdot)$ es un cuerpo.
\end{theorem}

La demostración de este teorema requiere verificar los axiomas AS1--AS4 y AM1--AM5. El punto crucial es que cuando $p$
es primo, todo elemento no nulo tiene inverso multiplicativo, lo cual se puede demostrar usando el algoritmo extendido
de Euclides.

\subsection{Implementación computacional en Python}

Una de las características más elegantes de Python es la \textbf{sobrecarga de operadores} (\textit{operator
    overloading}), que permite redefinir el comportamiento de operadores como \texttt{+}, \texttt{-}, \texttt{*} para tipos
de datos personalizados. Esto nos permite escribir código que se lee de forma natural, como si estuviéramos trabajando
con números ordinarios.

La conexión con los axiomas de cuerpo es directa: cuando definimos una clase que representa elementos de un cuerpo, los
métodos especiales de Python corresponden exactamente a las operaciones algebraicas (ver
Cuadro~\ref{tab:python-operators}).

\begin{table}[ht]
    \centering
    \begin{tabular}{lll}
        \toprule
        \textbf{Axioma} & \textbf{Operación}   & \textbf{Método Python}                 \\
        \midrule
        AS1--AS4        & Suma $a + b$         & \texttt{\_\_add\_\_(self, other)}      \\
                        & Negativo $-a$        & \texttt{\_\_neg\_\_(self)}             \\
        AM1--AM5        & Producto $a \cdot b$ & \texttt{\_\_mul\_\_(self, other)}      \\
                        & Inverso $a^{-1}$     & \texttt{\_\_invert\_\_(self)} o método \\
        \bottomrule
    \end{tabular}
    \caption{Correspondencia entre axiomas de cuerpo y métodos especiales de Python.}
    \label{tab:python-operators}
\end{table}

A continuación presentamos una implementación de $\ZZ/n\ZZ$ que ilustra estos conceptos:

\codefile[Python]{Implementación de aritmética modular en Python}{code/zmod.py}

Con esta implementación, podemos trabajar con aritmética modular de forma natural:

\codefile[Python]{Ejemplo de uso de la clase Zmod}{code/zmod_example.py}

\begin{exercise}[Verificador de axiomas para $\ZZ/n\ZZ$]
    Extienda la clase \texttt{Zmod} presentada anteriormente para incluir un método de clase \texttt{verify\_field\_axioms(n)} que:
    \begin{enumerate}
        \item Genere todos los elementos de $\ZZ/n\ZZ$.
        \item Verifique sistemáticamente cada uno de los axiomas AS1--AS4 y AM1--AM5.
        \item Retorne \texttt{True} si $\ZZ/n\ZZ$ es un cuerpo, \texttt{False} en caso contrario.
        \item Imprima cuál axioma falla cuando $n$ no es primo (por ejemplo, para $n = 6$, el elemento $2$ no tiene inverso
              multiplicativo).
    \end{enumerate}
    Pruebe su implementación con $n = 2, 3, 4, 5, 6, 7$ y verifique que solo los valores primos producen cuerpos.
\end{exercise}

\section{Axioma para la relación de orden en $\RR$}
Los números reales también están ordenados, lo que significa que existe una relación de orden que nos permite comparar
dos números reales y determinar cuál es mayor, menor o si son iguales. Esta relación de orden se define mediante los
siguientes axiomas:

\begin{definition}[Relación de orden en $\RR$]
    La relación de orden en los números reales es una relación binaria denotada por $<$ que satisface los siguientes axiomas:
    \begin{enumerate}
        \item[AO1.] \textbf{Tricotomía:} Para todo $a, b \in \RR$, exactamente una de las siguientes afirmaciones es verdadera: $a <
                  b$, $a = b$, o $a > b$.
        \item[AO2.] \textbf{Compatibilidad con la suma y el producto:} Si $a$ y $b$ son números reales tales que $0 < a$ y $0 < b$,
              entonces $0 < a + b$ y $0 < a \cdot b$.
        \item[AO3.] \textbf{Caracterización del orden:} Para $a$ y $b$ en $\RR$, $a < b$ si y solo si $b - a > 0$.
    \end{enumerate}
\end{definition}

Si $a < b$ también escribimos $b > a$. La relación $\leq$ se define por $a \leq b$ si y solo si $a < b$ o $a = b$. Los
axiomas AO1 y AO3 dicen que $<$ es un orden lineal, y AO2 relaciona el orden $<$ con las operaciones $+$ y $\cdot$.

Los axiomas anteriores nos dicen que $(\RR, +, \cdot, <)$ es un \textbf{cuerpo ordenado}. Las consecuencias de estos
axiomas son válidas para cualquier cuerpo ordenado, como por ejemplo los números racionales $\QQ$. Lo que distingue a
$\RR$ de otros cuerpos ordenados es el \textbf{axioma de completitud}, que enunciaremos más adelante.

\begin{definition}[Valor absoluto]
    Para cualquier $a \in \RR$, el \textbf{módulo} o \textbf{valor absoluto} $|a|$ se define como:
    \begin{equation*}
        |a| = \begin{cases}
            a,  & \text{si } a \geq 0, \\
            -a, & \text{si } a < 0.
        \end{cases}
    \end{equation*}
\end{definition}

\begin{exercise}[Propiedades del orden]
    Usando los axiomas AS1--AS4, AM1--AM5 y AO1--AO3, demuestre las siguientes propiedades:
    \begin{enumerate}
        \item Para cualesquiera $a, b, c \in \RR$: $a < b$ y $b < c$ implican $a < c$; $a \leq b$ y $b \leq c$ implican $a \leq c$.
              Es decir, las relaciones $<$ y $\leq$ son transitivas. La relación $\leq$ es reflexiva ($a \leq a$) mientras que $<$ no
              lo es.
        \item Si $a$ y $b$ están en $\RR$, $a \leq b$ y $b \leq a$, entonces $a = b$.
        \item Si $a, b, c \in \RR$ y $a > b$, entonces $a + c > b + c$; si $c > 0$ entonces $ac > bc$; si $c < 0$ entonces $ac < bc$.
        \item Si $a \in \RR$ y $a \neq 0$, entonces $a \cdot a > 0$. En consecuencia, $1 > 0$.
        \item Para cualquier $a \in \RR$, $a > 0$ si y solo si $-a < 0$.
        \item Si $a \in \RR$ y $a > 0$, entonces $a^{-1} > 0$; si $a < 0$, entonces $a^{-1} < 0$.
        \item Para cualesquiera $a, b, c \in \RR$, $a - b < a - c$ si y solo si $b > c$.
        \item Si $a$ y $b$ están en $\RR$, $a > b > 0$ implica $0 < a^{-1} < b^{-1}$, mientras que $a < b < 0$ implica $b^{-1} <
                  a^{-1} < 0$.
        \item Para cualesquiera $a$ y $b$ en $\RR$, con $a > 0$ y $b > 0$, $a > b \Leftrightarrow a \cdot a > b \cdot b$.
    \end{enumerate}
\end{exercise}

\begin{exercise}[Propiedades del valor absoluto]
    Demuestre las siguientes propiedades del valor absoluto:
    \begin{enumerate}
        \item Si $a \in \RR$, entonces $|-a| = |a|$. Además, $|a| = 0$ si y solo si $a = 0$.
        \item Para cualesquiera $a$ y $b$ en $\RR$: $|ab| = |a||b|$, $|a + b| \leq |a| + |b|$ (desigualdad triangular) y $|a - b|
                  \geq ||a| - |b||$.
        \item Si $a$ y $b$ están en $\RR$ y $b > 0$, entonces $|a| < b$ si y solo si $-b < a < b$.
    \end{enumerate}
\end{exercise}

\begin{exercise}[Orden en cuerpos finitos]
    Demuestre que para el sistema de dos elementos $\{\alpha, \beta\}$ definido en el Ejercicio~1.2, no es posible definir una relación de orden que satisfaga los axiomas AO1--AO3.
\end{exercise}

\begin{exercise}
    Considere el conjunto descrito en el Ejercicio~1.3 (números de la forma $a + b\sqrt{6}$). ¿Satisface este conjunto los axiomas para la relación de orden?
\end{exercise}

\begin{exercise}[Implementación de un cuerpo ordenado en Python]
    Implemente una clase \texttt{QuadraticExtension} para representar números de la forma $a + b\sqrt{6}$ donde $a, b \in \QQ$ (use la biblioteca \texttt{fractions}). La clase debe incluir:

    \textbf{Operaciones de cuerpo:}
    \begin{center}
        \begin{tabular}{ll}
            \toprule
            \textbf{Operación} & \textbf{Método Python}            \\
            \midrule
            $x + y$            & \texttt{\_\_add\_\_(self, other)} \\
            $-x$               & \texttt{\_\_neg\_\_(self)}        \\
            $x \cdot y$        & \texttt{\_\_mul\_\_(self, other)} \\
            $x^{-1}$           & \texttt{inverse(self)}            \\
            \bottomrule
        \end{tabular}
    \end{center}

    \textbf{Operaciones de orden:}
    \begin{center}
        \begin{tabular}{ll}
            \toprule
            \textbf{Operador} & \textbf{Método Python}           \\
            \midrule
            $x < y$           & \texttt{\_\_lt\_\_(self, other)} \\
            $x \leq y$        & \texttt{\_\_le\_\_(self, other)} \\
            $x > y$           & \texttt{\_\_gt\_\_(self, other)} \\
            $x \geq y$        & \texttt{\_\_ge\_\_(self, other)} \\
            \bottomrule
        \end{tabular}
    \end{center}

    \textit{Sugerencia:} Para comparar $a + b\sqrt{6}$ con $0$, considere que $\sqrt{6} \approx 2.449$. Si $b \neq 0$, puede determinar el signo analizando los casos según el signo de $b$ y comparando $a^2$ con $6b^2$.
\end{exercise}

\begin{exercise}[Verificación computacional de propiedades del orden]
    Implemente una función \texttt{verify\_order\_properties()} que, dados tres números racionales $a$, $b$ y $c$ (usando la biblioteca \texttt{fractions}), verifique computacionalmente las propiedades del Ejercicio~1.6:
    \begin{enumerate}
        \item Transitividad: si $a < b$ y $b < c$, entonces $a < c$.
        \item Si $a \neq 0$, entonces $a \cdot a > 0$.
        \item Si $a > 0$, entonces $-a < 0$.
        \item Si $a > 0$, entonces $a^{-1} > 0$.
    \end{enumerate}
    Pruebe su función con varios casos, incluyendo números negativos y fracciones.
\end{exercise}

\begin{hardproblem}[No completitud de $\QQ$]
    Una sucesión $(a_n)$ en un cuerpo ordenado se llama \textbf{sucesión de Cauchy} si para todo $\varepsilon > 0$ existe $N \in \NN$ tal que $|a_n - a_m| < \varepsilon$ para todo $n, m > N$. Un cuerpo ordenado se dice \textbf{completo} si toda sucesión de Cauchy converge a un elemento del cuerpo. El conjunto de los números racionales $\QQ$ satisface todos los axiomas de cuerpo (AS1--AS4 y AM1--AM5), sin embargo no es completo. Demuestre esto construyendo una sucesión de racionales $(r_n)$ tal que:
    \begin{enumerate}
        \item $r_n^2 < 2$ para todo $n \in \NN$.
        \item La sucesión $(r_n)$ es estrictamente creciente y acotada superiormente.
        \item $(r_n)$ es una sucesión de Cauchy.
    \end{enumerate}
    Concluya que $(r_n)$ no puede converger a ningún número racional, demostrando así que $\QQ$ no es completo.

    \textit{Sugerencia:} Considere la sucesión definida por $r_1 = 1$ y $r_{n+1} = \frac{1}{2}\left(r_n + \frac{2}{r_n}\right)$.
\end{hardproblem}

\section{Los números naturales}
En esta sección introducimos y estudiaremos algunas propiedades de los números naturales como subconjunto de $\RR$.
Intuitivamente pensamos en los números naturales como el conjunto que contiene al número $1$ y a todos los números que
se obtienen sumando $1$ repetidamente:

\begin{equation}
    \{1, 1+1, 1+1+1, \dots\}
\end{equation}

de modo que $1$ es un natural y si $n$ es un natural, entonces $n+1$ es un natural. Sin embargo, esta definición no es
lo suficientemente rigurosa para trabajar con los números naturales en el contexto de los números reales. Por lo tanto,
es necesario definir formalmente qué entendemos por «número natural».

\begin{definition}[Conjuntos inductivos]
    Un subconjunto $S \subseteq \RR$ es un \textbf{conjunto inductivo} si satisface las siguientes propiedades:
    \begin{enumerate}
        \item $1 \in S$.
        \item Si $n \in S$, entonces $n+1 \in S$.
    \end{enumerate}
\end{definition}

Como ejemplos tenemos que $\RR$ es un conjunto inductivo y también lo es ${x \in \RR \mid x \geq 1}$. Es fácil
verificar que la intersección de cualquier colección de conjunto inductivos es un conjunto inductivo. Sea $\mathcal{I}$
la colección de todos los conjuntos inductivos. Entonces, el conjunto de los números naturales se define como

\begin{equation}
    \NN = \bigcap_{S \in \mathcal{I}} S
\end{equation}

es decir,

\begin{equation}
    \NN = \{n \in \RR \mid n \in S \text{ para todo } S \in \mathcal{I}\}.
\end{equation}

Los elementos de $\NN$ son los números naturales (o enteros positivos). Si $A$ es un conjunto inductivo, entonces $A
    \supseteq \NN$. Además $1 \in \NN$ y si $n \in \NN$, entonces $n+1 \in \NN$, lo cual coincide con nuestra intuición
inicial.

\begin{theorem}[Principio de inducción matemática]
    Si $S \subseteq \NN$ es un conjunto inductivo, entonces $S = \NN$.
\end{theorem}

\begin{proof}
    Si $S$ es inductivo entonces $\NN \subseteq S$. Por hipótesis $S \subseteq \NN$. Por lo tanto, $S = \NN$.
\end{proof}

Veamos la forma como se utiliza el principio de inducción matemática. Supongamos que queremos demostrar que una
propiedad $P(n)$ es cierta para todo número natural $n$. Entonces, debemos demostrar que $P(1)$ es cierta y que si
$P(n)$ es cierta, entonces $P(n+1)$ es cierta. Si logramos demostrar ambas cosas, entonces $P(n)$ es cierta para todo
número natural $n$. En efecto, sea $S = \{n \in \NN \mid P(n) \text{ es cierta}\}$. Entonces $1 \in S$ y si $n \in S$,
entonces $P(n)$ es cierta, por lo que $P(n+1)$ es cierta, de modo que $n+1 \in S$. Por lo tanto, $S$ es un conjunto
inductivo y por el Teorema~1.1, $S = \NN$.

\begin{exercise}[Suma de los primeros $n$ números naturales]
    Demuestre por inducción matemática que para todo $n \in \NN$, se tiene que
    \begin{equation}
        \sum_{k=1}^{n} k = \frac{n(n+1)}{2}
    \end{equation}
\end{exercise}

\begin{exercise}[Suma de los primeros $n$ números impares]
    Demuestre por inducción matemática que para todo $n \in \NN$, se tiene que
    \begin{equation}
        \sum_{k=1}^{n} (2k-1) = n^2
    \end{equation}
\end{exercise}

\begin{exercise}[Suma de los primeros $n$ cubos]
    Demuestre por inducción matemática que para todo $n \in \NN$, se tiene que
    \begin{equation}
        \sum_{k=1}^{n} k^3 = \left(\frac{n(n+1)}{2}\right)^2
    \end{equation}
\end{exercise}

\begin{exercise}[Suma de los primeros $n$ números pares]
    Demuestre por inducción matemática que para todo $n \in \NN$, se tiene que
    \begin{equation}
        \sum_{k=1}^{n} 2k = n(n+1)
    \end{equation}
\end{exercise}

\begin{exercise}[Suma de los primeros $n$ números impares]
    Demuestre por inducción matemática que para todo $n \in \NN$, se tiene que
    \begin{equation}
        \sum_{k=1}^{n} (2k-1) = n^2
    \end{equation}
\end{exercise}

\begin{theorem}
    Si $n\in \NN$ y $n \neq 1$, entonces $n-1\in \NN$.
\end{theorem}

\begin{proof}
    Usamos inducción, sea $A = \{1\} \cup \{n \in \NN \mid n-1 \in \NN\}$.
    Claramente $1 \in A$. Supongamos que $k \in A$, entonces $(k+1)-1 = k \in A \subset \NN$, de modo que $k+1 \in A$.
    Por el principio de inducción matemática, $A = \NN$.
\end{proof}