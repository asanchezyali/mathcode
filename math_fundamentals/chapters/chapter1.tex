\chapter{Los números reales y sus propiedades}

En esta sección introducimos el sistema de los números reales y presentamos sus propiedades fundamentales. Como es bien
conocido, los números reales forma la base para el análisis matemático. Desde los griegos ya se sabía que los números
racionales, es decir, cocientes de números enteros, no eran suficientes para describir todas las cantidades
geométricas. Por ejemplo, en un triángulo reactángulo con catetos de longitud $1$, por el teorema de Pitágoras, la
longitud $c$ de la hipotenusa es tal que $c^2 = 2$, es fácil probar que no existe ningún número racional $p/q$ (con $p,
    q \in \ZZ$ y $q \neq 0$) que satisfaga esta ecuación. En efecto, si $c = p/q$ fuera una solución racional y tal que $p$
y $q$ no tuvieran factores primos en común, entonces $p^2 = 2q^2$ implica que $p^2$ es par, y por lo tanto $p$ es par.
Si $p$ es par, entonces existe un entero $k$ tal que $p = 2k$, y por lo tanto $p^2 = 4k^2$. Sustituyendo en la ecuación
original obtenemos $4k^2 = 2q^2$, o lo que es lo mismo, $q^2 = 2k^2$. Esto implica que $q^2$ es par, y por lo tanto $q$
es par. Pero si $p$ y $q$ son ambos pares, entonces tienen al menos el factor primo $2$ en común, lo cual contradice
nuestra suposición inicial. Por lo tanto, no existe ningún número racional que satisfaga la ecuación $c^2 = 2$. Sin
embargo el número la ecuación $c^2 = 2$ tiene una solución en los números reales, y a este número se denota por
$\sqrt{2}$. Para resolver este problema fue necesario ampliar el sistema de los número racionales a un sistema más
grande, el sistema de los números reales.

\section{Axiomas de suma}
Comenzaremos por postular que existe un conjunto $\RR$, cuyos elementos llamaremos \textbf{números reales}, junto con
las operaciones de suma y multiplicación y una relación de orden.

\begin{definition}[Suma en $\RR$]
    La \textbf{suma} en los números reales es una función
    \begin{equation*}
        + : \RR \times \RR \longrightarrow \RR,
    \end{equation*}
    que satisface los siguientes axiomas:
    \begin{enumerate}
        \item[AS1.] \textbf{Conmutatividad:} Para todo $a, b \in \RR$, se cumple que $a + b = b + a$.
        \item[AS2.] \textbf{Asociatividad:} Para todo $a, b, c \in \RR$, se cumple que $(a + b) + c = a + (b + c)$.
        \item[AS3.] \textbf{Elemento neutro:} Existe un elemento $0 \in \RR$ tal que para todo $a \in \RR$, se cumple que $a + 0 =
                  a$.
        \item[AS4.] \textbf{Elemento inverso:} Para todo $a \in \RR$, existe un elemento $-a \in \RR$ tal que $a + (-a) = 0$.
    \end{enumerate}
\end{definition}

Con estos cuatro axiomas, el conjunto $\RR$ junto con la operación de suma es un \textbf{grupo abeliano}. Además, a
partir de estos axiomas se pueden demostrar las siguientes propiedades adicionales de la suma en $\RR$.
\begin{theorem}
    Para todo $a, b, c \in \RR$, se cumplen las siguientes propiedades:
    \begin{enumerate}
        \item $a + (-a) = 0$.
        \item $-(-a) = a$.
        \item Si $a + b = a + c$, entonces $b = c$.
        \item $0 + a = a$.
        \item Si para algún $a \in \RR$ se cumple que $a + b = a$, entonces $b = 0$.
    \end{enumerate}
\end{theorem}

\section{Axiomas de multiplicación}

\begin{definition}[Multiplicación en $\RR$]
    La \textbf{multiplicación} en los números reales es una función
    \begin{equation*}
        \cdot : \RR \times \RR \longrightarrow \RR,
    \end{equation*}
    que satisface los siguientes axiomas:
    \begin{enumerate}
        \item[AM1.] \textbf{Conmutatividad:} Para todo $a, b \in \RR$, se cumple que $a \cdot b = b \cdot a$.
        \item[AM2.] \textbf{Asociatividad:} Para todo $a, b, c \in \RR$, se cumple que $(a \cdot b) \cdot c = a \cdot (b \cdot c)$.
        \item[AM3.] \textbf{Elemento neutro:} Existe un elemento $1 \in \RR \setminus \{0\}$ tal que para todo $a \in \RR$, se cumple
              que $1 \cdot a = a$.
        \item[AM4.] \textbf{Elemento inverso:} Para todo $a \in \RR \setminus \{0\}$, existe un elemento $a^{-1} \in \RR \setminus
                  \{0\}$ tal que $a \cdot a^{-1} = 1$.
        \item[AM5.] \textbf{Distributividad:} Para todo $a, b, c \in \RR$, se cumple que $a \cdot (b + c) = a \cdot b + a \cdot c$.
    \end{enumerate}
\end{definition}

Se puede observar que los axiomas AM1 a AM4 son similares a los axiomas AS1 a AS4, y juntos implican que el conjunto
$\RR \setminus \{0\}$ junto con la operación de multiplicación es también un grupo abeliano. El axioma AM5 expresa una
relación entre las operaciones de suma y multiplicación.

\begin{theorem}
    Para todo $a, b, c \in \RR$, se cumplen las siguientes propiedades:
    \begin{enumerate}
        \item $a \cdot 0 = 0$.
        \item Si $a \cdot b = 0$, entonces $a = 0$ o $b = 0$.
        \item Si $a \cdot b = a \cdot c$ y $a \neq 0$, entonces $b = c$.
        \item $a \cdot 1 = a$.
        \item Si para algún $a \in \RR \setminus \{0\}$ se cumple que $a \cdot b = a$, entonces $b = 1$.
    \end{enumerate}
\end{theorem}

Es posible que el lector se haya dado cuenta que, en la definición de las operaciones de suma y multiplicación en los
axiomas, no se utilizan las notaciones $+(a, b)$ y $\cdot(a, b)$ que tiene más sentido desde el punto de vista de
funcional. Esto se debe a que en matemáticas es común usar una notación infija para las operaciones binarias: la suma
$+(a, b)$ se escribe como $a + b$, y el producto $\cdot(a, b)$ se escribe como $a \cdot b$ o simplemente $ab$.

Por otro lado, el conjunto de los números reales junto con las operaciones de suma y multiplicación que hemos definido
hasta ahora se denomina un \textbf{cuerpo} y es una estructura algebraica fundamental en matemáticas que usualmente se
denota como $(\RR, +, \cdot)$. En general, si un conjunto $K$ junto con dos operaciones binarias $+$ y $\cdot$
satisface los axiomas AS1 a AS4 y AM1 a AM5, entonces decimos que $(K, +, \cdot)$ es un cuerpo.

\begin{exercise}
    Considere un sistema con dos elementos $\alpha$ y $\beta$ y las siguientes reglas de suma y multiplicación:
    \begin{align*}
        \alpha + \alpha     & = \alpha, & \alpha + \beta     & = \beta,  & \beta + \alpha     & = \beta,  & \beta + \beta     & = \alpha, \\
        \alpha \cdot \alpha & = \alpha, & \alpha \cdot \beta & = \alpha, & \beta \cdot \alpha & = \alpha, & \beta \cdot \beta & = \beta.
    \end{align*}
    Demuestre que este sistema forma un cuerpo.
\end{exercise}

\begin{exercise}
    Considere los números de la forma $a + b\sqrt{6}$ donde $a$ y $b$ son racionales. ¿Satisface este conjunto los axiomas de un cuerpo?
\end{exercise}

\begin{exercise}[Aproximación racional de $\sqrt{2}$]
    Como vimos, $\sqrt{2}$ no es un número racional. Sin embargo, podemos aproximarlo mediante fracciones. El método babilónico (o de Herón) genera una sucesión de aproximaciones racionales: dado $x_0 > 0$, se define
    \begin{equation*}
        x_{n+1} = \frac{1}{2}\left(x_n + \frac{2}{x_n}\right).
    \end{equation*}
    Implemente en Python una función que, dado un valor inicial $x_0$ y un número de iteraciones $n$, retorne la aproximación $x_n$ de $\sqrt{2}$. Utilice la biblioteca \texttt{fractions} para trabajar con números racionales exactos y observe cómo las fracciones se vuelven cada vez más complejas mientras se acercan a $\sqrt{2}$.
\end{exercise}

\begin{exercise}[Verificación de axiomas de cuerpo]
    Implemente en Python una clase \texttt{TwoElementField} que represente el cuerpo con dos elementos $\{\alpha, \beta\}$ del Ejercicio 1.1. La clase debe:
    \begin{enumerate}
        \item Definir las operaciones de suma y multiplicación según las tablas dadas.
        \item Incluir un método que verifique automáticamente cada uno de los axiomas AS1--AS4 y AM1--AM5 para todos los elementos
              del cuerpo.
        \item Identificar cuál elemento actúa como el $0$ (neutro aditivo) y cuál como el $1$ (neutro multiplicativo).
    \end{enumerate}
\end{exercise}

\begin{hardproblem}[No completitud de $\QQ$]
    Una sucesión $(a_n)$ en un cuerpo ordenado se llama \textbf{sucesión de Cauchy} si para todo $\varepsilon > 0$ existe $N \in \NN$ tal que $|a_n - a_m| < \varepsilon$ para todo $n, m > N$. Un cuerpo ordenado se dice \textbf{completo} si toda sucesión de Cauchy converge a un elemento del cuerpo. El conjunto de los números racionales $\QQ$ satisface todos los axiomas de cuerpo (AS1--AS4 y AM1--AM5), sin embargo no es completo. Demuestre esto construyendo una sucesión de racionales $(r_n)$ tal que:
    \begin{enumerate}
        \item $r_n^2 < 2$ para todo $n \in \NN$.
        \item La sucesión $(r_n)$ es estrictamente creciente y acotada superiormente.
        \item $(r_n)$ es una sucesión de Cauchy.
    \end{enumerate}
    Concluya que $(r_n)$ no puede converger a ningún número racional, demostrando así que $\QQ$ no es completo.

    \textit{Sugerencia:} Considere la sucesión definida por $r_1 = 1$ y $r_{n+1} = \frac{1}{2}\left(r_n + \frac{2}{r_n}\right)$.
\end{hardproblem}