\chapter{Cálculo en una variable}

Estas notas comienzan con una revisión rápida de ideas del cálculo de una variable. El objetivo es sentar las bases
para el desarrollo del análisis en varias variables.

\section{El sistema de los números reales}

Asumimos que existe un \textbf{sistema de números reales}, el cual es un conjunto $\RR$ que contiene dos elementos
distintos $0$ y $1$ y está dotado de las operaciones algebraicas de adición,

\begin{equation*}
    + : \RR \times \RR \longrightarrow \RR,
\end{equation*}

y multiplicación,

\begin{equation*}
    \cdot : \RR \times \RR \longrightarrow \RR.
\end{equation*}

La suma $+(a, b)$ se escribe como $a + b$, y el producto $\cdot(a, b)$ se escribe como $a \cdot b$ o simplemente $ab$.

\begin{theorem}[Axiomas de campo para $(\RR, +, \cdot)$]
    El sistema de números reales, con sus elementos distintos $0$ y $1$ y con sus operaciones de adición y multiplicación, satisface el siguiente conjunto de axiomas:
    \begin{enumerate}
        \item \textit{La adición es asociativa:} $(x + y) + z = x + (y + z)$ para todo $x, y, z \in \RR$.
        \item \textit{$0$ es una identidad aditiva:} $0 + x = x$ para todo $x \in \RR$.
        \item \textit{Existencia de inversos aditivos:} Para cada $x \in \RR$ existe $y \in \RR$ tal que $y + x = 0$.
        \item \textit{La adición es conmutativa:} $x + y = y + x$ para todo $x, y \in \RR$.
        \item \textit{La multiplicación es asociativa:} $x(yz) = (xy)z$ para todo $x, y, z \in \RR$.
        \item \textit{$1$ es una identidad multiplicativa:} $1x = x$ para todo $x \in \RR$.
        \item \textit{Existencia de inversos multiplicativos:} Para cada $x \in \RR$ no nulo existe $y \in \RR$ tal que $yx = 1$.
        \item \textit{La multiplicación es conmutativa:} $xy = yx$ para todo $x, y \in \RR$.
        \item \textit{La multiplicación distribuye sobre la adición:} $(x+y)z = xz + yz$ para todo $x, y, z \in \RR$.
    \end{enumerate}
\end{theorem}

Todo el álgebra básica de los números reales se deriva de los axiomas de campo. Los inversos aditivos y multiplicativos
son únicos, la ley de cancelación se cumple, $0 \cdot x = 0$ para todo número real $x$, y así sucesivamente.

El inverso aditivo de un número real $x$ se denota por $-x$, y la resta se define en términos de la adición y el
inverso aditivo como
\begin{equation*}
    - : \RR \times \RR \longrightarrow \RR, \qquad x - y = x + (-y) \quad \text{para todo } x, y \in \RR.
\end{equation*}

También asumimos que $\RR$ es un campo \textbf{ordenado}. Es decir, asumimos que existe un subconjunto $\RR^+$ de $\RR$
(los elementos \textbf{positivos}) tal que se cumplen los siguientes axiomas.

\begin{theorem}[Axiomas de Orden]
    \begin{enumerate}
        \item \textit{Axioma de tricotomía:} Para todo número real $x$, exactamente una de las siguientes condiciones se cumple:
              \begin{equation*}
                  x \in \RR^+, \qquad -x \in \RR^+, \qquad x = 0.
              \end{equation*}
        \item \textit{Clausura de los positivos bajo la adición:} Para todos los números reales $x$ e $y$, si $x \in \RR^+$ e $y \in \RR^+$ entonces también $x + y \in \RR^+$.
        \item \textit{Clausura de los positivos bajo la multiplicación:} Para todos los números reales $x$ e $y$, si $x \in \RR^+$ e $y \in \RR^+$ entonces también $xy \in \RR^+$.
    \end{enumerate}
\end{theorem}

Para todos los números reales $x$ e $y$, definimos
\begin{equation*}
    x < y
\end{equation*}
para significar
\begin{equation*}
    y - x \in \RR^+.
\end{equation*}

Las reglas usuales para desigualdades se derivan entonces de los axiomas.

\begin{theorem}[Completitud como criterio de búsqueda binaria]
    Toda secuencia de búsqueda binaria en el sistema de números reales converge a un límite único.
\end{theorem}

La convergencia es un concepto de análisis, y por lo tanto también lo es la completitud. Otra versión de la completitud
se expresa en términos de cotas de conjuntos.

\begin{theorem}[Completitud como criterio de cota superior]
    Todo subconjunto no vacío de $\RR$ que está acotado superiormente tiene una cota superior mínima.
\end{theorem}

Ambas formulaciones de completitud son enunciados de existencia.

\begin{definition}
    Un subconjunto $S$ de $\RR$ es \textbf{inductivo} si:
    \begin{enumerate}
        \item[\textbf{(i1)}] $0 \in S$,
        \item[\textbf{(i2)}] Para todo $x \in \RR$, si $x \in S$ entonces $x + 1 \in S$.
    \end{enumerate}
\end{definition}

Cualquier intersección de subconjuntos inductivos de $\RR$ es nuevamente inductiva. El conjunto de los \textbf{números
    naturales}, denotado $\NN$, es la intersección de todos los subconjuntos inductivos de $\RR$, es decir, $\NN$ es el
subconjunto inductivo más pequeño de $\RR$. No hay ningún número natural entre $0$ y $1$ (porque si lo hubiera,
eliminarlo de $\NN$ dejaría un subconjunto inductivo más pequeño de $\RR$), y así
\begin{equation*}
    \NN = \{0, 1, 2, \cdots\}.
\end{equation*}

\begin{theorem}[Teorema de Inducción]
    Sea $P(n)$ una forma proposicional definida sobre $\NN$. Supongamos que:
    \begin{itemize}
        \item $P(0)$ se cumple.
        \item Para todo $n \in \NN$, si $P(n)$ se cumple entonces también se cumple $P(n+1)$.
    \end{itemize}
    Entonces $P(n)$ se cumple para todos los números naturales $n$.
\end{theorem}

De hecho, las hipótesis del teorema dicen que $P(n)$ se cumple para un subconjunto de $\NN$ que es inductivo, y por lo
tanto el teorema se sigue de la definición de $\NN$ como el subconjunto inductivo más pequeño de $\RR$.

El \textbf{Principio Arquimediano} establece que el subconjunto $\NN$ de $\RR$ no está acotado superiormente.
Equivalentemente, la secuencia $\{1, 1/2, 1/3, \cdots\}$ converge a $0$. El Principio Arquimediano se sigue de la
suposición de que $\RR$ es completo en el sentido de secuencias de búsqueda binaria o en el sentido de cotas de
conjuntos. Una tercera versión de la completitud se expresa en términos de secuencias monótonas. Nuevamente es un
enunciado de existencia.

\begin{theorem}[Completitud como criterio de secuencia monótona]
    Toda secuencia monótona acotada en $\RR$ converge a un límite único.
\end{theorem}

Esta versión de la completitud se sigue de cualquiera de las otras dos. Sin embargo, no implica las otras dos a menos
que también asumamos el Principio Arquimediano.

El conjunto de los \textbf{enteros}, denotado $\ZZ$, es la unión de los números naturales y sus inversos aditivos,
\begin{equation*}
    \ZZ = \{0, \pm 1, \pm 2, \cdots\}.
\end{equation*}

