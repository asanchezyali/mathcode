% preamble.tex - Contains all packages and configurations
\documentclass[12pt, a4paper, oneside]{book}
%%% BASIC PACKAGES
\usepackage[T1]{fontenc}
\usepackage[utf8]{inputenc}
\usepackage{lmodern}
\usepackage[spanish, es-noshorthands]{babel}
\usepackage{comment}
\usepackage{textcomp}

%%% DOCUMENT GEOMETRY
\usepackage[top=2.5cm, bottom=2.5cm, right=2cm, left=2cm, headheight=15pt, marginparwidth=2.5cm]{geometry}
\usepackage{multicol}
\usepackage{lscape}

%%% MATH PACKAGES
\usepackage{amsmath, amsthm, amsfonts, latexsym, cancel}
\usepackage{amsbsy}
\usepackage{thmtools}
\usepackage{amssymb}
\usepackage{mathtools}

% Display full-size fractions in inline math
\everymath{\displaystyle}

%%% STYLE AND DESIGN PACKAGES
\usepackage[dvipsnames]{xcolor}
\usepackage{tikz}
\usetikzlibrary{calc,shapes,shadows,decorations.pathmorphing,matrix,positioning}
\usepackage{pgfplots}
\pgfplotsset{compat=1.18}
\usepackage{tikz-qtree}
\usepackage{mdframed}

%%% CODE LISTINGS
\usepackage{listings}
\usepackage{tcolorbox}
\tcbuselibrary{listings, skins, breakable}
\renewcommand{\lstlistingname}{Código}

% Modern code color palette
\definecolor{codeBackground}{RGB}{250, 250, 252}
\definecolor{codeFrame}{RGB}{218, 220, 235}
\definecolor{codeAccent}{RGB}{99, 102, 241}
\definecolor{codeHeader}{RGB}{75, 85, 120}
\definecolor{codeKeyword}{RGB}{124, 58, 237}
\definecolor{codeString}{RGB}{5, 150, 105}
\definecolor{codeComment}{RGB}{148, 163, 184}
\definecolor{codeNumber}{RGB}{180, 190, 205}
\definecolor{codeFunction}{RGB}{37, 99, 235}
\definecolor{codeVariable}{RGB}{217, 70, 239}

\lstset{
basicstyle=\ttfamily\fontsize{8}{9.5}\selectfont,
keywordstyle=\color{codeKeyword}\bfseries,
keywordstyle=[2]\color{codeFunction},
commentstyle=\color{codeComment}\itshape,
stringstyle=\color{codeString},
numbers=left,
numberstyle=\tiny\color{codeNumber}\ttfamily,
numbersep=10pt,
breaklines=true,
breakatwhitespace=true,
showstringspaces=false,
tabsize=4,
extendedchars=true,
inputencoding=utf8,
literate={á}{{\'a}}1 {é}{{\'e}}1 {í}{{\'i}}1 {ó}{{\'o}}1 {ú}{{\'u}}1
{ñ}{{\~n}}1 {Á}{{\'A}}1 {É}{{\'E}}1 {Í}{{\'I}}1 {Ó}{{\'O}}1
{Ú}{{\'U}}1 {Ñ}{{\~N}}1
}

% Counter for code listings
\newcounter{codelisting}[chapter]
\renewcommand{\thecodelisting}{\thechapter.\arabic{codelisting}}

% Custom tcolorbox style for code listings (con header estilo teorema)
\newtcblisting{codebox}[2][]{
    listing only,
    listing options={language=#2},
    enhanced,
    breakable,
    colback=codeBackground,
    colframe=codeFrame,
    boxrule=0.4pt,
    left=10pt,
    right=10pt,
    top=10pt,
    bottom=10pt,
    arc=5pt,
    before skip=\topskip,
    after skip=\topskip,
    #1
}

% Style for lstinputlisting with tcolorbox (basado en theorembase)
\tcbset{
    codestyle/.style={
            listing only,
            enhanced,
            breakable,
            colback=codeBackground,
            colframe=codeFrame,
            colbacktitle=codeHeader,
            coltitle=white,
            boxrule=0.4pt,
            arc=5pt,
            left=25pt,
            right=10pt,
            top=5pt,
            bottom=5pt,
            toptitle=5pt,
            bottomtitle=5pt,
            lefttitle=10pt,
            fonttitle=\bfseries,
            before skip=\topskip,
            after skip=\topskip,
        }
}

% \codefile[language]{caption}{file}
\newcommand{\codefile}[3][Python]{%
    \refstepcounter{codelisting}%
    \begin{tcolorbox}[codestyle,
            title={Código~\thecodelisting~--~#2}]
        \lstinputlisting[language=#1]{#3}
    \end{tcolorbox}%
}

%%% BOOK-SPECIFIC PACKAGES
\usepackage{fancyhdr}
\usepackage{booktabs}
\usepackage{minitoc}
\usepackage{epigraph}
\setlength{\epigraphwidth}{0.6\textwidth}
\usepackage{titlesec}

%%% ALGORITHMS
\usepackage[ruled,vlined,spanish,onelanguage]{algorithm2e}
\SetKwInput{KwEntrada}{Entrada}
\SetKwInput{KwSalida}{Salida}
\SetKwFor{ParaCada}{para cada}{hacer}{fin}
\SetKwIF{Si}{SinoSi}{Sino}{si}{entonces}{sino si}{sino}{fin}
\SetKwFor{Para}{para}{hacer}{fin}
\SetKw{Retornar}{retornar}

%%% BIBLIOGRAPHY AND REFERENCES
\usepackage[backend=biber, style=apa, sorting=ynt]{biblatex}
\usepackage{csquotes}
\usepackage{fontawesome5}

% Format URL as clickable globe icon
\DeclareFieldFormat{url}{\href{#1}{\faGlobe}}
\DeclareFieldFormat{urldate}{}

% Add spacing between bibliography entries
\setlength{\bibitemsep}{1em}

% Remove hanging indent and add square bullets
\setlength{\bibhang}{0pt}
\defbibenvironment{bibliography}
{\list
    {\tiny$\blacksquare$}
    {\setlength{\leftmargin}{1.5em}%
        \setlength{\itemindent}{0pt}%
        \setlength{\itemsep}{\bibitemsep}%
        \setlength{\parsep}{\bibparsep}}}
{\endlist}
{\item}

%%% INDEX AND GLOSSARY
\usepackage{imakeidx}
\usepackage[toc]{glossaries}
\makeindex[columns=3, title=Índice Alfabético]

%%% GRAPHICS AND FIGURES
\usepackage{graphicx}
\usepackage{caption}
\usepackage{subcaption}
\graphicspath{{./images/}}

%%% HYPERLINKS
\usepackage[
    breaklinks=true,
    colorlinks=true,
    linkcolor=blue,
    citecolor=blue,
    urlcolor=gray
]{hyperref}
\usepackage{xurl}

%%% COLOR DEFINITIONS
\definecolor{theoremBG}{RGB}{235,245,255}
\definecolor{theoremBorder}{RGB}{210,225,250}
\definecolor{theoremHeader}{RGB}{70,100,140}
\definecolor{definitionBG}{RGB}{235,255,240}
\definecolor{definitionBorder}{RGB}{210,245,220}
\definecolor{definitionHeader}{RGB}{60,110,80}
\definecolor{remarkBG}{RGB}{255,240,230}
\definecolor{remarkBorder}{RGB}{250,220,205}
\definecolor{remarkHeader}{RGB}{140,90,60}
\definecolor{easyBG}{RGB}{230,242,255}
\definecolor{easyBorder}{RGB}{205,220,250}
\definecolor{easyHeader}{RGB}{70,100,140}
\definecolor{mediumBG}{RGB}{255,240,225}
\definecolor{mediumBorder}{RGB}{250,220,205}
\definecolor{mediumHeader}{RGB}{140,90,60}
\definecolor{hardBG}{RGB}{255,230,240}
\definecolor{hardBorder}{RGB}{250,205,220}
\definecolor{hardHeader}{RGB}{140,70,90}
\definecolor{exerciseBG}{RGB}{240,240,255}
\definecolor{exerciseBorder}{RGB}{220,220,250}
\definecolor{exerciseHeader}{RGB}{90,90,130}
\definecolor{exampleBG}{RGB}{250,250,235}
\definecolor{exampleBorder}{RGB}{230,230,200}
\definecolor{exampleHeader}{RGB}{100,100,70}

%%% THEOREM STYLES WITH TCOLORBOX
% Base style for all theorem-like environments
\tcbset{
    theorembase/.style={
            enhanced,
            breakable,
            boxrule=0.4pt,
            arc=5pt,
            left=10pt,
            right=10pt,
            top=10pt,
            bottom=10pt,
            toptitle=5pt,
            bottomtitle=5pt,
            fonttitle=\bfseries,
            before skip=\topskip,
            after skip=\topskip,
        }
}

%%% THEOREM COUNTERS
\newcounter{theorem}[chapter]
\newcounter{corollary}[theorem]
\newcounter{lemma}[chapter]
\newcounter{definition}[chapter]
\newcounter{remark}[chapter]
\newcounter{example}[chapter]
\newcounter{exercise}[chapter]
\newcounter{easyproblem}[chapter]
\newcounter{mediumproblem}[chapter]
\newcounter{hardproblem}[chapter]

\renewcommand{\thetheorem}{\thechapter.\arabic{theorem}}
\renewcommand{\thecorollary}{\thetheorem.\arabic{corollary}}
\renewcommand{\thelemma}{\thechapter.\arabic{lemma}}
\renewcommand{\thedefinition}{\thechapter.\arabic{definition}}
\renewcommand{\theremark}{\thechapter.\arabic{remark}}
\renewcommand{\theexample}{\thechapter.\arabic{example}}
\renewcommand{\theexercise}{\thechapter.\arabic{exercise}}
\renewcommand{\theeasyproblem}{\thechapter.\arabic{easyproblem}}
\renewcommand{\themediumproblem}{\thechapter.\arabic{mediumproblem}}
\renewcommand{\thehardproblem}{\thechapter.\arabic{hardproblem}}

%%% THEOREM ENVIRONMENTS
% Theorem
\newenvironment{theorem}[1][]{%
    \refstepcounter{theorem}%
    \begin{tcolorbox}[theorembase,
            colback=theoremBG!40,
            colframe=theoremBorder,
            colbacktitle=theoremHeader,
            coltitle=white,
            title={Teorema~\thetheorem\ifx&#1&\else{} -- #1\fi}]%
        }{%
    \end{tcolorbox}%
}

% Corollary
\newenvironment{corollary}[1][]{%
    \refstepcounter{corollary}%
    \begin{tcolorbox}[theorembase,
            colback=theoremBG!40,
            colframe=theoremBorder,
            colbacktitle=theoremHeader,
            coltitle=white,
            title={Corolario~\thecorollary\ifx&#1&\else{} -- #1\fi}]%
        }{%
    \end{tcolorbox}%
}

% Lemma
\newenvironment{lemma}[1][]{%
    \refstepcounter{lemma}%
    \begin{tcolorbox}[theorembase,
            colback=theoremBG!40,
            colframe=theoremBorder,
            colbacktitle=theoremHeader,
            coltitle=white,
            title={Lema~\thelemma\ifx&#1&\else{} -- #1\fi}]%
        }{%
    \end{tcolorbox}%
}

% Definition
\newenvironment{definition}[1][]{%
    \refstepcounter{definition}%
    \begin{tcolorbox}[theorembase,
            colback=definitionBG!40,
            colframe=definitionBorder,
            colbacktitle=definitionHeader,
            coltitle=white,
            title={Definición~\thedefinition\ifx&#1&\else{} -- #1\fi}]%
        }{%
    \end{tcolorbox}%
}

% Remark
\newenvironment{remark}[1][]{%
    \refstepcounter{remark}%
    \begin{tcolorbox}[theorembase,
            colback=remarkBG!40,
            colframe=remarkBorder,
            colbacktitle=remarkHeader,
            coltitle=white,
            title={Observación~\theremark\ifx&#1&\else{} -- #1\fi}]%
        }{%
    \end{tcolorbox}%
}

% Example
\newenvironment{example}[1][]{%
    \refstepcounter{example}%
    \begin{tcolorbox}[theorembase,
            colback=exampleBG!40,
            colframe=exampleBorder,
            colbacktitle=exampleHeader,
            coltitle=white,
            title={Ejemplo~\theexample\ifx&#1&\else{} -- #1\fi}]%
        }{%
    \end{tcolorbox}%
}

% Exercise (sin caja)
\newenvironment{exercise}[1][]{%
    \refstepcounter{exercise}%
    \par\vspace{0.5\baselineskip}\noindent%
    \textbf{Ejercicio~\theexercise\ifx&#1&\else{} -- #1\fi.}\space%
}{%
    \par\vspace{0.5\baselineskip}%
}

% Easy problem
\newenvironment{easyproblem}[1][]{%
    \refstepcounter{easyproblem}%
    \begin{tcolorbox}[theorembase,
            colback=easyBG!40,
            colframe=easyBorder,
            colbacktitle=easyHeader,
            coltitle=white,
            title={Problema~\theeasyproblem\ifx&#1&\else{} -- #1\fi}]%
        }{%
    \end{tcolorbox}%
}

% Medium problem
\newenvironment{mediumproblem}[1][]{%
    \refstepcounter{mediumproblem}%
    \begin{tcolorbox}[theorembase,
            colback=mediumBG!40,
            colframe=mediumBorder,
            colbacktitle=mediumHeader,
            coltitle=white,
            title={Problema~\themediumproblem\ifx&#1&\else{} -- #1\fi}]%
        }{%
    \end{tcolorbox}%
}

% Hard problem
\newenvironment{hardproblem}[1][]{%
    \refstepcounter{hardproblem}%
    \begin{tcolorbox}[theorembase,
            colback=hardBG!40,
            colframe=hardBorder,
            colbacktitle=hardHeader,
            coltitle=white,
            title={Problema~\thehardproblem\ifx&#1&\else{} -- #1\fi}]%
        }{%
    \end{tcolorbox}%
}

%%% PROOF STYLE
\renewenvironment{proof}[1][\proofname]
{\par\noindent\textit{#1.}\space}
{\hfill$\square$\par\vspace{\baselineskip}}

%%% CUSTOM CHAPTER FORMATTING
\titleformat{\chapter}[display]
{\normalfont\huge\bfseries}
{\chaptertitlename\ \thechapter}
{20pt}
{\Huge}
\titlespacing*{\chapter}{0pt}{50pt}{40pt}

%%% CUSTOM SECTION FORMATTING
\titleformat{\section}
{\normalfont\Large\bfseries}
{\thesection}
{1em}
{}
\titlespacing*{\section}{0pt}{3.5ex plus 1ex minus .2ex}{2.3ex plus .2ex}

%%% HEADER AND FOOTER SETTINGS
\pagestyle{fancy}
\fancyhf{}
\renewcommand{\chaptermark}[1]{\markboth{\chaptername\ \thechapter.\ #1}{}}
\renewcommand{\sectionmark}[1]{\markright{\thesection.\ #1}}
\fancyhead[LE,RO]{\thepage}
\fancyhead[LO]{\textit{\rightmark}}
\fancyhead[RE]{\textit{\leftmark}}
\renewcommand{\headrulewidth}{0.4pt}
\renewcommand{\footrulewidth}{0pt}

%%% MATH OPERATORS
\DeclareMathOperator{\sen}{sen\,}
\DeclareMathOperator{\senh}{senh\,}
\DeclareMathOperator{\sech}{sech\,}
\DeclareMathOperator{\rot}{rot\,}
\DeclareMathOperator{\dive}{div\,}
\DeclareMathOperator{\nulo}{nul\,}
\DeclareMathOperator{\im}{img}
\DeclareMathOperator{\id}{id}
\DeclareMathOperator{\Tr}{Tr\,}
\DeclareMathOperator{\Crit}{Crit\,}
\DeclareMathOperator{\md}{\;mod}
\DeclareMathOperator{\ord}{ord}
\DeclareMathOperator{\mcd}{mcd}
\DeclareMathOperator{\mcm}{mcm}

%%% MATH SYMBOLS AND COMMANDS
\newcommand{\KK}{\mathbb{K}}
\newcommand{\EE}{\mathcal{E}}
\newcommand{\FF}{\mathbb{F}}
\newcommand{\HH}{\mathbb{H}}
\newcommand{\RR}{\mathbb{R}}
\newcommand{\ZZ}{\mathbb{Z}}
\newcommand{\NN}{\mathbb{N}}
\newcommand{\NNO}{\mathbb{N}_{_0}}
\newcommand{\QQ}{\mathbb{Q}}
\newcommand{\CC}{\mathbb{C}}

%%% PARAGRAPH SETTINGS
\setlength{\parindent}{0cm}
\setlength{\parskip}{2mm}

%%% ORPHANS AND WIDOWS PREVENTION
\clubpenalty=1000
\widowpenalty=1000
\displaywidowpenalty=1000

% --- Custom commands for title page ---
\makeatletter
\newcommand{\printbooktitle}{\@title}
\newcommand{\printbookauthor}{\@author}
\newcommand{\printbookdate}{\@date}
\makeatother
\let\cleardoublepage\clearpage
\addbibresource{bibliography/references.bib}
